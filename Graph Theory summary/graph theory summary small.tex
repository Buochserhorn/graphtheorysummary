\documentclass[11pt, fleqn, a4paper, landscape]{article}
\usepackage{listings}
\usepackage{color}
\usepackage{datetime}
\usepackage[margin=2cm]{geometry}
\usepackage[utf8]{inputenc}
\usepackage{textcomp}
\usepackage{amssymb, etoolbox}

\usepackage{amsmath, amsthm}
\usepackage{mathtools}

\setlength{\mathindent}{0pt} 
\setlength{\arraycolsep}{2pt} %reducing whitespace in matrices in small fonts

\usepackage{pgfplots}
	\pgfplotsset{
		compat=1.12
	}
\allowdisplaybreaks
\usepackage[compact]{titlesec} 
\titlespacing{\section}{0pt}{0pt}{0pt} % this and the following, are for spaces around the mentioned parts
\titlespacing{\subsection}{0pt}{0pt}{0pt}
\titlespacing{\subsubsection}{0pt}{0pt}{0pt}
\titlespacing{\paragraph}{0pt}{0pt}{0pt}
%\newcommand{\sectionbreak}{\clearpage} %start a new section on a new page

\usepackage{multicol} % multicolumn layout

\setlength\parindent{0pt} %Do not indent after an empty line


%\renewcommand{\baselinestretch}{1.5} %Zeilenabstand von 1.5
\usepackage[shortlabels]{enumitem} % das und folgendes für Abstand zwischen items [shortlabels um i),ii) etc. zu erlauben]
\setlist[enumerate]{nolistsep}
\setlist[itemize]{nolistsep} % depends on [shortlabels]{enumitem}
	%indentation of enumerate and itemize, numebrs indicate the level of nestedness, depends on usepackage{enumitem}
	\setlist[enumerate,1]{leftmargin=.6cm}
	\setlist[enumerate,2]{leftmargin=.6cm}
	\setlist[enumerate,3]{leftmargin=.6cm}
	\setlist[enumerate,4]{leftmargin=.6cm}
	\setlist[itemize,1]{leftmargin=.6cm}
	\setlist[itemize,2]{leftmargin=.6cm}
	\setlist[itemize,3]{leftmargin=.6cm}
	\setlist[itemize,4]{leftmargin=.6cm}

\makeatletter
\@addtoreset{problem}{subsection}
\makeatother

\newtheoremstyle{plain}% name of the style to be used
  {}   % ABOVESPACE
  {\parskip}   % BELOWSPACE
  {\itshape}  % BODYFONT
  {0pt}       % INDENT (empty value is the same as 0pt)
  {\bfseries} % HEADFONT
  {.}         % punctuation between head and body
  {5pt plus 1pt minus 1pt} % space after theorem head; " " = normal interword space
  {}          % CUSTOM-HEAD-SPEC
  
\newtheoremstyle{definition}
  {}% measure of space to leave above the theorem. E.g.: 3pt
  {\parskip}% measure of space to leave below the theorem. E.g.: 3pt
  {}% name of font to use in the body of the theorem
  {0pt}% measure of space to indent
  {\bfseries}% name of head font
  {.}% punctuation between head and body
  {5pt plus 1pt minus 1pt}% space after theorem head; " " = normal interword space
  {}% Manually specify head

\theoremstyle{plain} % default, italic, extra space
\newtheorem{thm}{Theorem}
\newtheorem{lem}[thm]{Lemma}
\newtheorem{pro}[thm]{Proposition}
\newtheorem{cor}[thm]{Corollary}

\theoremstyle{remark} % roman, no space
\newtheorem{rem}[thm]{Remark}
\newtheorem{nota}[thm]{Notation}
\newtheorem{claim}[thm]{Claim}
\newtheorem{fact}{Fact}
\newtheorem{que}[thm]{Question}
\newtheorem{problem}{Problem}

\theoremstyle{definition} % roman, extra spave
\newtheorem{defi}[thm]{Definition}


% Modify theorem counter to match that of the section unit
\preto{\section}{\renewcommand{\thethm}{\thesection.\arabic{thm}}}%


% Reset the counter at every sectional unit
\makeatletter
\@addtoreset{thm}{section}
\makeatother

\begin{document}

\setlength{\abovedisplayskip}{0pt}
\setlength{\belowdisplayskip}{0pt}
\setlength{\abovedisplayshortskip}{0pt}
\setlength{\belowdisplayshortskip}{0pt}

\begin{multicols}{3}
%\begin{tiny}

\section{Basic notions}
\subsection{Graphs}
\begin{defi}
A graph $G$ is a pair $G = (V,E)$ where $V$ is a set of vertices and $E$ is a (multi)set of unordered pairs of vertices. The elements of $E$ are called edges. We write $V (G)$ for the set of vertices and $E(G)$ for the set of edges of a graph $G$. Also, $|G| = |V (G)|$ denotes the number of
vertices and $e(G) = |E(G)|$ denotes the number of edges.
\end{defi}
\begin{defi}
A loop is an edge $(v, v)$ for some $v\in V$ . An edge $e = (u, v)$ is a multiple edge if it appears multiple times in $E$. A graph is simple if it has no loops or multiple edges.
\end{defi}
\begin{defi}
\begin{itemize}
\item Vertices $u, v$ are adjacent in $G$ if $(u, v)\in E(G)$.
\item An edge $e \in E(G)$ is incident to a vertex $v \in V (G)$ if $v \in e$.
\item Edges $e, e'$ are incident if $e\cap e'\ne \emptyset$.
\item If $(u, v) \in E $ then $v$ is a neighbour of $u$.
\end{itemize}
\end{defi}
\addtocounter{thm}{1}

\subsection{Graph isomorphism}
\addtocounter{thm}{1}
\begin{defi}
 Let $G_1 = (V_1,E_1)$ and $G_2 = (V_2,E_2)$ be graphs. An isomorphism $\phi: G_1 \to G_2$ is a bijection (a one-to-one correspondence) from $V_1$ to $V_2$ such that $(u, v) \in E_1$ if and only if $(\phi(u), \phi(v))\in E_2$. We say $G_1$ is isomorphic to $G_2$ if there is an isomorphism between them.
\end{defi}
\addtocounter{thm}{1}
\begin{rem}
Isomorphism is an equivalence relation of graphs. This means that
\begin{itemize}
\item Any graph is isomorphic to itself
\item If $G_1$ is isomorphic to $G_2$ then $G_2$ is isomorphic to $G_1$
\item If $G_1 $ is isomorphic to $G_2$ and $G_2$ is isomorphic to $G_3$, then $G_1$ is isomorphic to $G_3$.
\end{itemize}
\end{rem}


\begin{defi}
An unlabelled graph is an isomorphism class of graphs.
\end{defi} 

\subsection{The adjacency and incidence matrices}

Let $[n] = \{1, \dots , n\}.$

\begin{defi}
Let $G = (V,E)$ be a graph with $V = [n].$ The adjacency matrix $A = A(G)$ is the graph with $ V = [n]$. The adjacency matrix $A = A(G)$ is the
$n \times n $ symmetric matrix defined by
\[a_{ij}=\begin{cases}1 & if (i,j)\in E\\ 0 & otherwise\end{cases}\]
\end{defi} 
\addtocounter{thm}{1}
\begin{rem}
Any adjacency matrix $A $ is real and symmetric, hence the spectral theorem proves that $A$ has an orthogonal basis of eigenvalues with real eigenvectors. This important fact allows us to use spectral methods in graph theory. Indeed, there is a large subfield of graph theory called spectral graph theory.
\end{rem}

\begin{defi}
Let $G = (V,E)$ be a graph with $V = \{v_1, \dots , v_n\}$ and $E = \{e_1, \dots , e_m\}$. Then the incidence matrix $B = B(G)$ of $G$ is the $n\times m$ matrix defined by \[b_{ij}=\begin{cases}1 & if v_i\in e_j\\ 0 & otherwise\end{cases}\]
\end{defi}
\addtocounter{thm}{1}
\begin{rem}
Every column of $B$ has $|e| = 2$ entries 1.
\end{rem}

\subsection{Degree}

\begin{defi}
Given $G = (V,E)$ and a vertex $v \in V$ , we define the neighbourhood $N(v)$ of $v$ to
be the set of neighbours of $v$. Let the degree $d(v)$ of $v$ be $|N(v)|$, the number of neighbours of $v$. A vertex $v$ is isolated if $d(v) = 0.$
\end{defi}
\begin{rem}
$d(v)$ is the number of 1s in the row corresponding to v in the adjacency matrix $A(G)$ or the incidence matrix $B(G)$.
\end{rem} 
\addtocounter{thm}{1}
\begin{fact}
For any graph $G$ on the vertex set $[n]$ with adjacency and incidence matrices $A$ and $B$, we have $BB^T = D + A$, where
$
\begin{pmatrix}
d(1) & 0 & 0  \\
0 & \ddots & 0  \\
0 & 0 & d(n)  
\end{pmatrix}
$  
\end{fact}

\begin{nota}
The minimum degree of a graph $G$ is denoted $\delta(G)$, the maximum degree is denoted $\Delta(G)$. The average degree is \[\overline{d}(G) =\frac{\sum_{v\in G}}{|V(G)|}\] Note that $\delta\le\overline{d}\le\Delta$.
\end{nota}

\begin{defi}
A graph $G$ is $d$-regular if and only if all vertices have degree $d$.
\end{defi}
\addtocounter{thm}{1}
\begin{pro}
For every $G = (V,E)$, $\sum_{v\in G}d(G)=2|E|$
\end{pro}

\begin{cor}
Every graph has an even number of vertices of odd degree.
\end{cor}

\subsection{Subgraphs}

\begin{defi}
A graph $H = (U, F)$ is a subgraph of a graph $G = (V,E)$ if $U \subseteq V$ and $F \subseteq E$. If $U = V$ then $H$ is called spanning.
\end{defi}

\begin{defi}
Given $G = (V,E)$ and $U\subseteq  V (U \ne \emptyset)$, let $G[U]$ denote the graph with vertex set $U$ and edge set $E(G[U]) = \{e \in E(G) : e \subseteq U\}$. (We include all the edges of $G$ which have both
endpoints in $U$). Then $G[U]$ is called the subgraph of $G$ induced by $U$.
\end{defi}
\addtocounter{thm}{1}
\subsection{Special graphs}

\begin{itemize}
\item  $K_n$ is the complete graph, or a clique. Take $n$ vertices and all possible edges connecting them.
\item An empty graph has no edges.
\item $G = (V,E)$ is bipartite if there is a partition $V = V_1 \cup V_2$ into two disjoint sets such that each $e \in E(G)$ intersects both $V_1$ and $V_2$.
\item $K_{n,m}$ is the complete bipartite graph. Take $n + m$ vertices partitioned into a set $ A$ of size $n$ and a set $B$ of size $m$, and include every possible edge between $A$ and $B$.
\end{itemize}
\addtocounter{thm}{1}
\subsection{Walks, paths and cycles}

\begin{defi}
A walk in $G$ is a sequence of vertices $v_0, v_1, \dots , v_k$, and a sequence of edges $(v_i, v_{i+1}) \in E(G)$. A walk is a path if all $v_i$ are distinct. If for such a path with $k \ge 2$, $(v_0, v_k)$ is also
an edge in $G$, then $v_0, v_1 \dots , v_k, v_0$ is a cycle. For multigraphs, we also consider loops and pairs of multiple edges to be cycles.
\end{defi}

\begin{defi}
The length of a path, cycle or walk is the number of edges in it.
\end{defi}
\addtocounter{thm}{1}
\begin{pro}
Every walk from u to v in G contains a path between u and v.
\end{pro}

\begin{pro}
Every $G$ with minimum degree $ \delta\ge 2$ contains a path of length $\delta$ and a cycle of length at least $\delta + 1$.
\end{pro}

\begin{rem}
Note that we have also proved that a graph with minimum degree $\delta\ge 2$ contains cycles of at least $\delta-1$ different lengths. This fact, and the statement of Proposition 1.32, are both tight, to see this, consider the complete graph $G = K_{\delta+1}$.
\end{rem} 

\subsection{Connectivity}

\begin{defi}
A graph $G$ is connected if for all pairs $u, v \in G$, there is a path in $G$ from $u$ to $v$.
\end{defi}
Note that it suffices for there to be a walk from u to v, by Proposition 1.31.
\addtocounter{thm}{1}
\begin{defi}
A (connected) component of $G$ is a connected subgraph that is maximal by inclusion. We say $G$ is connected if and only if it has one connected component.
\end{defi}
\addtocounter{thm}{1}
\begin{pro}
A graph with $n$ vertices and $m$ edges has at least $n-m$ connected components.
\end{pro}

\subsection{Graph operations and parameters}

\begin{defi}
Given $G = (V,E)$, the complement $\overline{G}$ of $G$ has the same vertex set $V$ and $(u, v) \in E(\overline{G})$ if and only if $(u, v) \notin E(G)$.
\end{defi}
\addtocounter{thm}{1}
\begin{defi}
A clique in $G$ is a complete subgraph in $G$. An independent set is an empty
induced subgraph in $G$.
\end{defi}
\addtocounter{thm}{1}
\begin{nota}
Let $\omega(G)$ denote the number of vertices in a maximum-size clique in $G$, let $\alpha(G)$ denote the number of vertices in a maximum-size independent set in $G$.
\end{nota}. 

\begin{claim}
A vertex set $U\subseteq V (G)$ is a clique if and only if $U\subseteq V(\overline{G})$ is an independent set.
\end{claim}

\begin{cor}
We have $\omega(G)=\alpha(\overline{G})$ and $\alpha(G)=\omega(\overline{G})$.
\end{cor}

\section{Trees}
\subsection{Trees}
\begin{defi}
A graph having no cycle is acyclic. A forest is an acyclic graph, a tree is a connected acyclic graph. A leaf (or pendant vertex ) is a vertex of degree 1.
\end{defi}
\addtocounter{thm}{1}
\begin{lem}
Every finite tree with at least two vertices has at least two leaves. Deleting a leaf from an $n$-vertex tree produces a tree with $n-1$ vertices.
\end{lem}

\subsection{Equivalent definitions of trees}

\begin{thm}
For an n-vertex simple graph $G$ (with $n \ge 1$), the following are equivalent (and
characterize the trees with n vertices).
(a) G is connected and has no cycles.
(b) G is connected and has n - 1 edges.
(c) G has n - 1 edges and no cycles.
(d) For every pair u, v 2 V (G), there is exactly one u, v-path in G.
\end{thm}

\begin{defi}
An edge of a graph is a cut-edge if its deletion disconnects the graph.
\end{defi}

\begin{lem}
An edge contained in a cycle is not a cut-edge.
\end{lem}


\begin{defi}
Given a connected graph $G$, a spanning tree $T$ is a subgraph of $G$ which is a tree and contains every vertex of $G$.
\end{defi}

\begin{cor}
\begin{itemize}
\item  Every connected graph on n vertices has at least n - 1 edges and contains a spanning tree,
\item Every edge of a tree is a cut-edge,
\item Adding an edge to a tree creates exactly one cycle.
\end{itemize}
\end{cor}

\subsection{Cayley’s formula}
\addtocounter{thm}{1}\addtocounter{thm}{1}
\begin{thm}[Cayley’s Formula]
There are $n^{n-2}$ trees with vertex set $[n]$.
\end{thm}

\begin{defi}[Prüfer code]
Let $T$ be a tree on an ordered set $S$ of n vertices. To compute the
Prüfer sequence $f(T)$, iteratively delete the leaf with the smallest label and append the label of its neighbour to the sequence. After $n - 2$ iterations a single edge remains and we have produced a sequence $f(T)$ of length $n - 2$.
\end{defi}
\addtocounter{thm}{1}
\begin{pro}
For an ordered $n$-element set $S$, the Prüfer code $f$ is a bijection between the trees with vertex set $S$ and the sequences in $S^{n-2}$.
\end{pro}
\addtocounter{thm}{1}
\begin{defi}
A directed graph, or digraph for short, is a vertex set and an edge (multi-)set of ordered pairs of vertices. Equivalently, a digraph is a (possibly not-simple) graph where each edge is assigned a direction. The out-degree (respectively in-degree) of a vertex is the number of edges
incident to that vertex which point away from it (respectively, towards it).
\end{defi}

\section{Connectivity}
\subsection{Vertex connectivity}

\begin{defi}
A vertex cut in a connected graph $G = (V,E)$ is a set $ S \subseteq V$ such that $G\setminus S := G[V \setminus S]$ has more than one connected component. A cut vertex is a vertex $v$ such that $\{v\}$ is a cut.
\end{defi}
\begin{defi}
$G$ is called $k$-connected if $|V (G)|> k$ and if $G\setminus X$ is connected for every set $X \subseteq V$ with $|X|< k$. In other words, no two vertices of $G$ are separated by fewer than $k$ other vertices. Every (non-empty) graph is 0-connected and the 1-connected graphs are precisely the non-trivial connected graphs. The greatest integer $k$ such that $G$ is k-connected is the connectivity $\kappa (G)$ of $G$.
17
\item $G = K_n: \kappa (G) = n - 1$
\item $G = K_{m,n}, m \le n: \kappa (G) = m$. Indeed, let $G$ have bipartition $A \cup B$, with $|A|= m$  and $|B|= n$. Deleting $A$ disconnects the graph. On the other hand, deleting $S \subset V$ with $|S|< m$ leaves both $A\setminus S$ and $B\setminus S$ non-empty and any $a \in A\setminus S$ is connected to any $b \in B \setminus S$. Hence $G\setminus S$ is connected.
\end{defi}
\begin{pro}
For every graph $G$, $\kappa (G) \le \delta(G)$.
\end{pro} 

\begin{rem}
High minimum degree does not imply connectivity. Consider two disjoint copies of $K_n$.
\end{rem}
\begin{thm}[Mader 1972]
Every graph of average degree at least $4k$ has a $k$-connected subgraph.
\end{thm}

\subsection{Edge connectivity}
\begin{defi}
A disconnecting set of edges is a set $F \subseteq E(G)$ such that $G\setminus F$ has more than one component. Given $S$, $T \subset V (G)$, the notation $[S, T]$ specifies the set of edges having one endpoint
in $S$ and the other in $T$. An edge cut is an edge set of the form $[S, S]$, where $S$ is a non-empty proper subset of $V (G)$. A graph is $k$-edge-connected if every disconnecting set has at least $k$ edges.
The edge-connectivity of $G$, written $\kappa'(G)$, is the minimum size of a disconnecting set. One edge disconnecting G is called a bridge.
$G = Kn: \kappa'(G) = n - 1.$
\end{defi}
\addtocounter{thm}{1}
\begin{rem}
An edge cut is a disconnecting set but not the other way around. However, every minimal disconnecting set is a cut.
\end{rem}

\begin{thm}
$\kappa (G) \le \kappa'(G) \le \delta(G).$
\end{thm}

\subsection{Blocks}
\begin{defi}
A block of a graph $G$ is a maximal connected subgraph of $G$ that has no cut-vertex.
If $G$ itself is connected and has no cut-vertex, then $G$ is a block.
\end{defi}
\addtocounter{thm}{1}
\begin{rem}
If a block $B$ has at least three vertices, then $B$ is 2-connected. If an edge is a block of $G$ then it is a cut-edge of $G$.
\end{rem}

\begin{pro}
Two blocks in a graph share at most one vertex.
\end{pro}

\begin{defi}
The block graph of a graph $G$ is a bipartite graph $H$ in which one partite set consists of the cut-vertices of $G$, and the other has a vertex $b_i$ for each block $B_i$ of $G$. We include $(v, b_i)$ as an edge of $H$ if and only if $v \in B_i.$
\end{defi}
\addtocounter{thm}{1}
\begin{pro}
The block graph of a connected graph is a tree.
\end{pro}

\subsection{2-connected graphs}

\begin{defi}
Two paths are internally disjoint if neither contains a non-endpoint vertex of the other. We denote the length of the shortest path from $u$ to $v$ (the distance from $u$ to $v$) by $d(u, v)$.
\end{defi}

\begin{thm}[Whitney 1932]
A graph $G$ having at least three vertices is 2-connected if and only
if each pair $u, v \in V (G)$ is connected by a pair of internally disjoint $u, v$-paths in $G$.
\end{thm}

\begin{cor}
$G$ is 2-connected and $|G|\ge 3$ if and only if every two vertices in $G$ lie on a common cycle.
\end{cor}
\subsection{Menger’s Theorem}

\begin{defi}
Let $A,B \subseteq V$ . An $A-B$ path is a path with one endpoint in $A$, the other endpoint in $B$, and all interior vertices outside of $A \cup B$. Any vertex in $A - B$ is a trivial $A-B$ path.

If $X \subseteq V$ (or $X \subseteq E$) is such that every $A-B$ path in $G$ contains a vertex (or an edge) from $X$, we say that $X$ separates the sets $A$ and $B$ in $G$. This implies in particular that $A \cap B \subseteq X$.
\end{defi}
\begin{thm}[Menger 1927]
Let $G = (V,E)$ be a graph and let $S, T \subseteq V$ . Then the maximum
number of vertex-disjoint $S-T$ paths is equal to the minimum size of an $S-T$ separating vertex set.
\end{thm}

\begin{cor}
For $S \subseteq V$ and $v\in V \setminus S$, the minimum number of vertices distinct from $v$ separating $v$ from $S$ in $G$ is equal to the maximum number of paths forming an $v-S$ fan in $G$. (that is, the maximum number of $\{v\}-S$ paths which are disjoint except at $v$).
\end{cor}

\begin{defi}
The line graph of $G$, written $L(G)$, is the graph whose vertices are the edges of $G$, with $(e, f) \in E(L(G))$ when $e = (u, v)$ and $f = (v,w)$ in $G$ (i.e. when $e$ and $f$ share a vertex).
\end{defi}
\addtocounter{thm}{1}
\begin{cor}
Let u and v be two distinct vertices of G.
\begin{enumerate}
\item If $(u, v) \notin E$, then the minimum number of vertices different from $u, v$ separating $u$ from $v$ in $G$ is equal to the maximum number of internally vertex-disjoint $u-v$ paths in $G$.
\item The minimum number of edges separating $u$ from v in G is equal to the maximum number of edge-disjoint $u-v$ paths in $G$.
\end{enumerate}
\end{cor}

\begin{thm}[Global Version of Menger’s Theorem]
\begin{enumerate}
\item A graph is k-connected if and only if it contains k internally vertex-disjoint paths between any
two vertices.
\item A graph is k-edge-connected if and only if it contains k edge-disjoint paths between any two
vertices.
\end{enumerate}
\end{thm}

\section{Eulerian and Hamiltonian cycles}
\subsection{Eulerian trails and tours}
\addtocounter{thm}{1}
\begin{defi}
A trail is a walk with no repeated edges.
\end{defi}

\begin{defi}
An Eulerian trail in a (multi)graph $G = (V,E)$ is a walk in $G$ passing through every edge exactly once. If this walk is closed (starts and ends at the same vertex) it is called an Eulerian tour.
\end{defi}
\addtocounter{thm}{1}
\begin{thm}
A connected (multi)graph has an Eulerian tour if and only if each vertex has even degree.
\end{thm}

\begin{lem}
Every maximal trail in an even graph (i.e., a graph where all the vertices have even degree) is a closed trail.
\end{lem} 

\begin{cor}
A connected multigraph $G$ has an Eulerian trail if and only if it has either 0 or 2 vertices of odd degree.
\end{cor}

\subsection{Hamilton paths and cycles}
\begin{defi}
A Hamilton path/cycle in a graph $G$ is a path/cycle visiting every vertex of $G$ exactly once. A graph $G$ is called Hamiltonian if it contains a Hamilton cycle.
\end{defi}
\addtocounter{thm}{1}
\begin{pro}
If $G$ is Hamiltonian then for any set $S \subseteq V$ the graph $G\setminus S$ has at most $|S|$ connected components.
\end{pro}

\begin{cor}
If a connected bipartite graph $G = (V,E)$ with bipartition $V = A\cup B$ is Hamiltonian then $|A|=|B|$.
\end{cor}
\addtocounter{thm}{1}
\begin{thm}[Dirac 1952]
If $G$ is a simple graph with $n \ge 3$ vertices and if $\delta(G) \ge n/2$, then $G$ is Hamiltonian.
\end{thm}
\addtocounter{thm}{1}
\begin{thm}[Ore 1960]
If $G$ is a simple graph with $n\ge 3$ vertices such that for every pair of
non-adjacent vertices $u, v$ of $G$ we have $d(u) + d(v)\ge |G|$, then $G $ is Hamiltonian.
\end{thm}
\section{Matchings}
\begin{defi}
A set of edges $M \subseteq E(G)$ in a graph $G$ is called a matching if $e \cap e' = \emptyset$ for any pair of edges $e, e' \in M$.

A matching is perfect if $|M|=\frac{|V (G)|}{2}$ , i.e. it covers all vertices of $G$. We denote the size of the maximum matching in $G$, by $\nu(G)$.

$G=K_n; \nu(G)=\left\lfloor\frac{n}{2}\right\rfloor$

$G=K_{s,t}; s\le t, \nu(G)=s$

$\nu(Petersen Graph)=5$
\end{defi}
\addtocounter{thm}{1}
\begin{rem}
A matching in a graph $G$ corresponds to an independent set in the line graph $L(G)$.
\end{rem}
\begin{defi}
A set of vertices $T \subseteq V (G)$ of a graph  $G$ is called a cover of $G$ if every edge $e \in E(G)$ intersects $T (e \cap T \neq \emptyset)$, i.e., $G \setminus T$ is an empty graph. Then, $\tau(G)$ denotes the size of
the minimum cover.

$G = Kn: \tau(G) = n - 1$

$G = K_{s,t}, s \le t: \tau(G) = s$

$\tau(Petersen Graph)=6$
\end{defi}
\addtocounter{thm}{1}
\begin{pro}
$\nu(G) \le\tau(G) \le 2\nu(G).$
\end{pro}
\subsection{Real-world applications of matchings}

\subsection{Hall’s Theorem}
\begin{thm}[Hall 1935]
A bipartite graph $G = (V,E)$ with bipartition $V = A\cup B$ has a matching
covering $A$ if and only if $|N(S)|\ge|S|\forall S \subseteq A$
\end{thm}

\begin{cor}
 If in a bipartite graph $G = (A\cup B,E)$ we have $|N(S)|\ge|S|-d$ for every set $S \subseteq A$ and some fixed $d\in\mathbb{N}$, then $G$ contains a matching of cardinality $|A|- d$.
\end{cor}

\begin{cor}
If a bipartite graph $G = (A \cup B,E) $ is $k$-regular with $k \ge 1$, then $G$ has a perfect matching.
\end{cor}

\begin{cor}
Every regular graph of positive even degree has a 2-factor (a spanning 2-regular subgraph).
\end{cor}

\begin{rem}
A 2-factor is a disjoint union of cycles covering all the vertices of a graph
\end{rem}

\begin{defi}
Let $A_1, \dots ,A_n$ be a collection of sets. A family $\{a_1, \dots , a_n\}$  is called a system of distinct representatives (SDR) if all the $a_i$ are distinct, and $a_i \in A_i$ for all $i$.
\end{defi}

\begin{cor}
A collection $A_1, \dots ,A_n$ has an SDR if and only if for all $I \subseteq [n]$ we have $|\bigcup_{i\in I} A_i|\ge|I|$.
\end{cor}
\addtocounter{thm}{1}
\begin{thm}[König 1931]
If $G = (A \cup B,E)$ is a bipartite graph, then the maximum size of a
matching in $G$ equals the minimum size of a vertex cover of $G$.
\end{thm}

\subsection{Matchings in general graphs: Tutte’s Theorem}
Given a graph $G$, let $q(G)$ denote the number of its odd components, i.e. the ones of odd order. If G has a perfect matching then clearly
$q(G\setminus S) \le|S|for all S \subseteq V (G)$
since every odd component of GnS will send an edge of the matching to S, and each such edge covers a different vertex in S.

\begin{thm}[Tutte 1947]
A graph $G$ has a perfect matching if and only if $q(G\setminus S) \le|S|$ for
all $S \subseteq V (G)$.
\end{thm}

\begin{cor}[Petersen 1891]
Every 3-regular graph with no cut-edge has a perfect matching.
\end{cor}
\addtocounter{thm}{1}
\begin{cor}[Berge 1958]
The largest matching in an $n$-vertex graph $G$ covers $n+min_{S\subseteq V (G)}(|S|- q(G\setminus S))$ vertices.
\end{cor}

\section{Planar Graphs}
\begin{defi}
A polygonal path or polygonal curve in the plane is the union of many line segments such that each segment starts at the end of the previous one and no point appears in more than one segment except for common endpoints of consecutive segments. In a polygonal $u, v$-path, the beginning of the first segment is $u$ and the end of the last segment is $v$.

A drawing of a graph $G$ is a function that maps each vertex $v\in V (G)$ to a point $f(v)$ in the plane and each edge $uv$ to a polygonal $f(u), f(v)$-path in the plane. The images of vertices are distinct.
A point in $f(e)\cup f(e')$ other than a common end is a crossing. A graph is planar if it has a drawing without crossings. Such a drawing is a planar embedding of $G$. A plane graph is a particular drawing of a planar graph in the plane with no crossings.
\end{defi}
\addtocounter{thm}{1}
\begin{rem}
We get the same class of graphs if we only require images of edges to be continuous curves. This is because any continuous line can be arbitrarily accurately approximated by a polygonal curve.
\end{rem}

\begin{defi}
An open set in the plane is a set $U \subset \mathbb{R}^2$ such that for every $p\in U$, all points within some small distance from $p$ belong to $U$. A region is an open set $U$ that contains a polygonal $u, v$-path for every pair $u, v \in U$ (that is, it is “path-connected”). The faces of a plane graph are the maximal regions of the plane that are disjoint from the drawing.
\end{defi}

\begin{thm}[Jordan curve theorem]
A simple closed polygonal curve C consisting of finitely
many segments partitions the plane into exactly two faces, each having C as boundary.
\end{thm}

\begin{rem}
This is not true in three dimensions. In $\mathbb{R}$ there is a surface called the Möbius band which has only one side.
\end{rem}

\begin{rem}
The faces of $G$ are pairwise disjoint (they are separated by the edges of G). Two points are in the same face if and only if there is a polygonal path between them which does not cross an edge of $G$. Also, note that a finite graph has a single unbounded face (the area “outside” of the graph).
\end{rem}

\begin{pro}
A plane forest has exactly one face.
\end{pro}

\begin{defi}
The length of the face $f$ in a planar embedding of $G$ is the sum of the lengths of the walks in $G$ that bound it.
\end{defi}
\addtocounter{thm}{1}
\begin{pro}
If $l(f_i)$ denotes the length of a face $f_i$ in a plane graph $G$, then $2e(G) = \sum l(f_i)$.
\end{pro}

\begin{thm}[Euler's formula 1758]
If a connected plane graph $G$ has exactly $n$ vertices, $e$ edges
and $f$ faces, then $n - e + f = 2$.
\end{thm}

\begin{rem}
The fact that deleting an edge in a cycle decreases the number of faces by one can be proved formally using the Jordan curve theorem.
\end{rem}
\begin{thm}
If $G$ is a planar graph with at least three vertices, then $e(G) \le 3|G|- 6$. If $G$ is also triangle-free, then $e(G)\le 2jG|- 4$.
\end{thm}

\begin{cor}
If $G$ is a planar bipartite $n$-vertex graph with $n \ge 3$ vertices then $G$ has at most $2n - 4$ edges.
\end{cor}

\begin{cor}
$K_5$ and $K_{3,3}$ are not planar.
\end{cor}

\begin{rem}[Maximal planar graphs / triangulations]
The proof of Theorem 6.14 shows that
having 3n - 6 edges in a simple n-vertex planar graph requires 2e = 3f, meaning that every face is
a triangle. If G has some face that is not a triangle, then we can add an edge between non-adjacent
vertices on the boundary of this face to obtain a larger plane graph. Hence the simple plane graphs
with 3n - 6 edges, the triangulations, and the maximal plane graphs are all the same family.
\end{rem}

\subsection{Platonic Solids}

\begin{defi}
A polytope is a solid in 3 dimensions with flat faces, straight edges and sharp corners. Faces of a polytope are|oined at the edges. A polytope is convex if the line connecting any two points of the polytope lies inside the polytope.
\end{defi}
\addtocounter{thm}{1}
\begin{defi}
A regular or Platonic solid is a convex polytope which satisfies the following:
\begin{enumerate}
\item all of its faces are congruent regular polygons,
\item all vertices have the same number of faces adjacent to them.
\end{enumerate}
\end{defi}
 
\begin{cor}
If $K$ is a convex polytope with $v$ vertices, $e$ edges and $f$ faces then $v - e + f = 2$.
\end{cor}

\section{Graph colouring}
\subsection{Vertex colouring}

\begin{defi}
A $k$-colouring of $G$ is a labeling $f : V (G) \to \{1, \dots , k\}$. It is a proper $k$-colouring if $(x, y) \in E(G)$ implies $f(x) \ne f(y)$. A graph $G $ is $k$-colourable if it has a proper $k$-colouring. The
chromatic number $\chi(G)$ is the minimum $k$ such that $G$ is $k$-colourable. If $\chi(G) = k$, then $G$ is
$k$-chromatic. If $\chi(G) = k$, but $\chi(H) < k$ for every proper subgraph $H$ of $G$, then $G$ is colour-critical or $k$-critical.

$\chi(K_n) = n$
\end{defi}
\addtocounter{thm}{1}
\begin{rem}
The vertices having a given colour in a proper colouring must form an independent set, so $\chi(G)$ is the minimum number of independent sets needed to cover $V (G).$ Hence $G$ is $k$-colourable if and only if $G$ is $k$-partite. Multiple edges do not affect chromatic number. Although we define
$k$-colouring using numbers from $\{1, \dots , k\}$ as labels, the numerical values are usually unimportant, and we may use any set of size $k$ as labels.
\end{rem}

\subsection{Some motivation}
\addtocounter{thm}{1}\addtocounter{thm}{1}
\subsection{Simple bounds on the chromatic number}
\begin{claim}
If $H$ is a subgraph of $G$ then $\chi(H) \le \chi(G)$.
\end{claim}

\begin{cor}
$\chi(G) \ge \omega(G)$
\end{cor}
\addtocounter{thm}{1}
\begin{pro}
$\chi(G) \ge\frac{|V (G)|}{\alpha(G)}$
\end{pro}

\begin{claim}
For any graph $G = (V,E)$ and any $U \subseteq V$ we have $\chi(G) \le \chi(G[U]) + \chi(G[V \setminus U])$.
\end{claim}

\begin{claim}
For any graphs $G_1$ and $G_2$ on the same vertex set, $\chi(G_1 \cup G_2) \le \chi(G1)\chi(G2).$
\end{claim}


\begin{pro}
\begin{enumerate}[(i)]
\item $\chi(G)\chi(\overline{G})\ge |G|$
\item $\chi(G)+\chi(\overline{G})\le |G|+1$
\end{enumerate}
\end{pro}

\subsection{Greedy colouring}

\begin{defi}
The greedy colouring with respect to a vertex ordering $v_1, \dots , v_n$ of $V (G)$ is obtained by colouring vertices in the order $v_1, \dots , v_n$, assigning to $v_i$ the smallest-indexed colour not already used on its lower-indexed neighbours.
\end{defi}
\addtocounter{thm}{1}
\begin{defi}
Let $G = (V,E)$ be a graph. We say that $G$ is $k$-degenerate if every subgraph of $G$ has a vertex of degree less than or equal to $k$.
\end{defi}

\begin{pro}
$G$ is $k$-degenerate if and only if there is an ordering $v_1, \dots , v_n$ of the vertices of $G$ such that each $v_i$ has at most $k$ neighbours among the vertices $v_1, \dots , v_{i-1}$.
\end{pro}

\begin{defi}
Define $dg(G)$ to be the minimum $k$ such that $G$ is $k$-degenerate.
\end{defi}
\begin{rem}
$\delta(G) \le dg(G) \le \Delta(G)$.
\end{rem}

\begin{thm}
$\chi(G) \le 1 + dg(G)$
\end{thm}

\begin{cor}
$\chi(G) \le \Delta(G) + 1.$
\end{cor}

\begin{rem}
This bound is tight if $G = K_n$ or if $G$ is an odd cycle.
\end{rem}

\begin{thm}[Brooks 1941]
If $G$ is a connected graph other than a clique or an odd cycle, then
$\chi(G) \le \Delta(G).$
\end{thm}

\subsection{Colouring planar graphs}
\begin{claim}
A (simple) planar graph $G$ contains a vertex $v$ of degree at most 5.
\end{claim}

\begin{cor}
A planar graph $G$ is 5-degenerate and thus 6-colourable.
\end{cor}
\begin{thm}[5 colour theorem, Heawood 1890]
Every planar graph G is 5-colourable.
\end{thm}

\begin{thm}[Appel-Haken 1977, conjectured by Guthrie in 1852]
Every planar graph is 4-
colourable. (the countries of every plane map can be 4-coloured so that neighbouring countries get
distinct colours).
\end{thm}). 
\addtocounter{thm}{1}
\subsection{An application: the art gallery theorem}

\begin{thm}
For any museum with $n$ walls, $\left\lfloor n=3\right\rfloor$ guards suffice.
\end{thm}

\section{More colouring results}
\begin{thm}[Gallai, Roy]
If $D$ is an orientation of $G$ with longest path length $l(D)$, then $\chi(G) \le 1 + l(D)$. Furthermore, equality holds for some orientation of $G$.
\end{thm} 

\subsection{Large girth and large chromatic number}

The bound $\chi(G) \ge \omega(G)$ can be tight, but (surprisingly) it can also be arbitrarily bad. There are graphs having arbitrarily large chromatic number, even though they do not contain $K_3$. Many constructions of such graphs are known, though none are trivial. We give one here.
\addtocounter{thm}{1}
\begin{thm}
Mycielski’s construction produces a $(k + 1)$-chromatic triangle-free graph from a $k$-chromatic triangle-free graph.
\end{thm}

\begin{defi}
The girth of a graph is the length of its shortest cycle.
\end{defi}
\begin{thm}[Erdos 1959]
Given $k \ge 3$ and $g \ge 3$, there exists a graph with girth at least g and
chromatic number at least k.
\end{thm} 
\addtocounter{thm}{2}
\begin{thm}
There is a tournament on $n$ vertices where any $\frac{\log_2(n)}{2}$ vertices are beaten by some other vertex.
\end{thm}
\subsection{Chromatic number and clique minors}
\begin{defi}
Let $e = (x, y)$ be an edge of a graph $G = (V,E)$. By $G\slash e$ we denote the graph obtained from $G$ by contracting the edge $e$ into a new vertex $v_e$, which becomes adjacent to all the former neighbours of $x$ and of $y$. 

$H$ is a minor of $G$ if it can be obtained from G by deleting vertices and edges, and contracting edges.
\end{defi}
\addtocounter{thm}{2}
\begin{thm}[Mader]
If the average degree of $G$ is at least $2t-2$ then $G$ has a $K_t$ minor.
\end{thm}

\begin{rem}
It is known that $\overline{d}(G) \ge ct\sqrt{log(t)}$ already implies the existence of a $K_t$ minor in $G$, for some constant $c > 0$.
\end{rem}
\subsection{Edge-colourings}
\begin{defi}
A $k$-edge-colouring of $G$ is a labeling $f : E(G) \to [k]$, the labels are “colours”. A proper $k$-edge-colouring is a $k$-edge-colouring such that edges sharing a vertex receive different colours, equivalently, each colour class is a matching. A graph $G$ is $k$-edge-colourable if it has a
proper $k$-edge-colouring. The edge-chromatic number or chromatic index $\chi'(G)$ is the minimum $k$ such that $G$ is $k$-edge colourable.
\end{defi}

\begin{rem}
\begin{enumerate}[(i)]
\item An edge-colouring of a graph $G$ is the same as a vertex-colouring of its line graph $L(G)$.
\item A graph $G$ with maximum degree $\Delta$ has $\chi'(G) \ge \Delta$ since the edges incident to a vertex of degree $\Delta$ must have different colours.
\item If $G$ has maximum degree $\Delta$ then $L(G)$ has maximum degree at most $2(\Delta-1).\Rightarrow \chi'(G) \le 2\Delta - 1$ 
\end{enumerate}
\end{rem}

\begin{thm}[König 1916]
If $G$ is a bipartite multigraph, then $\chi'(G) = \Delta(G)$.
\end{thm}

\begin{thm}[Vizing]
Let $G$ be a simple graph with maximum degree $\Delta$. Then $\Delta(G)\le \chi0(G) \le \Delta(G) + 1.$
\end{thm}
\addtocounter{thm}{1}

\subsection{List colouring}

\begin{defi}
For each vertex $v$ in a graph $G$, let $L(v)$ denote a list of colours available for $v$. A list colouring or choice function from a given collection of lists is a proper colouring $f$ such that $f(v)$ is chosen from $L(v)$. A graph $G$ is $k$-choosable or $k$-list-colourable if it has a proper list colouring from every assignment of $k$-element lists to the vertices. The list chromatic number or choosability $\chi_l(G)$ is the minimum $k$ such that $G$ is $k$-choosable.
\end{defi}

\begin{thm}[Erdos, Rubin, Taylor 1979]
If $m = \binom{2k-1}{k}$, then $K_{m,m}$ is not $k$-choosable.
\end{thm}). 

\begin{defi}
Let $L(e)$ denote the list of colours available for $e$. A list edge-colouring is a proper edge-colouring $f$ with $f(e)$ chosen from $L(e)$ for each $e$. The list chromatic index or edge-choosability $\chi' l(G)$ is the minimum $k$ such that $G$ has a proper list edge-colouring for each assignment of lists of size $k$ to the edges. Equivalently, $\chi'
l(G) = \chi_l(L(G))$, where $L(G)$ is the line graph of $G$.
\end{defi}
\addtocounter{thm}{1}
\begin{thm}[Galvin 1995]
$\chi' l(K_{n,n}) = n$
\end{thm}.

\begin{defi}
A kernel of a digraph is an independent set $S$ having an edge to every vertex
outside $S$. A digraph is kernel-perfect if every induced sub-digraph has a kernel. Given a function $f : V (G) \to \mathbb{N}$, the graph $G$ is $f$-choosable if a proper list colouring can be chosen whenever the lists
satisfy $|L(x)|\ge f(x)$ for each $x$.
\end{defi}

\begin{lem}
If $D$ is a kernel-perfect orientation of $G$ and $f(x) = d^-_D(x)$ for all $x \in V (G)$, then $G$ is $(1 + f)$-choosable.
\end{lem}

\section{The Matrix Tree Theorem}

\begin{thm}[Cayley's formula]
There are $n^{n-2}$ labeled trees on $n$ vertices.
\end{thm}Cayley’s formula). 

Now consider an arbitrary connected simple graph $G$ on vertex set $[n]$, and denote the number of spanning trees by $t(G)$. The following celebrated result is Kirchhoff’s matrix tree theorem. To formulate it, consider the incidence matrix $B$ of $G$ (as in Definition 1.13), and replace one of the
two 1’s by -1 in an arbitrary manner to obtain the matrix $C$ (we say $C$ is the incidence matrix of an orientation of $G$). $ M = CC^T$ is then a symmetric $n \times n$ matrix, which is 

\[
\begin{pmatrix}
d(1) & 0 & 0  \\
0 & \ddots & 0  \\
0 & 0 & d(n)  
\end{pmatrix}-A_G\]

\begin{thm}[Matrix tree theorem]
We have $t(G) = det M_{ii}$ for all $i = 1, \dots , n$, where $M_{ii}$ results
from $M$ by deleting the $i$-th row and the $i$-th column.
\end{thm}

\begin{thm}[Binet,  Cauchy]
If $P$ is an $r \times s$ matrix and $Q$ is an $s \times r$ matrix with $r\le s$, then \[det(PQ) =\sum_Z (det P_Z)(detQ_Z)\]
where $P_Z$ is the $r\times r$ submatrix of $P$ with column set $Z$, and $Q_Z$ is the $r\times r$ submatrix of $Q$ with the corresponding rows $Z$, and the sum is over all $r$-sets $Z \subseteq [s]$.
\end{thm}

\subsection{Lattice paths and determinants}
see p.55 and following

\section{More Theorems on Hamiltonicity}
\begin{defi}
The (Hamiltonian) closure of a graph $G$, denoted $C(G)$, is the supergraph of $G$ on $V (G)$ obtained by iteratively adding edges between pairs of nonadjacent vertices whose degree sum is at least $n$, until no such pair remains.
\end{defi}
\begin{thm}[Bondy Chvátal 1976]
A simple $n$-vertex graph is Hamiltonian if and only if its closure is Hamiltonian.
\end{thm}
\begin{thm}[Chvátal 1972]
Suppose $G$ has vertex degrees $d_1\le \dots d_n$. If $i < n/2$ implies
that $d_i > i$ or $d_{n-i} \ge n - i$, then $G$ is Hamiltonian.
\end{thm}

\begin{thm}[Chvátal-Erdos 1972]
If $\kappa (G)\ge\alpha(G)$, then $G$ has a Hamiltonian cycle (unless
$G = K_2$).
\end{thm}

\subsection{Pósa’s Lemma}
Let $P$ be a path in a graph $G$, say from $u$ to $v$. Given a vertex $x \in P$, we write $x^-$ for the vertex preceding $x$ on $P$, and $x^+$ for the vertex following $x$ on $P$ (whenever these exist). Similarly, for $X \subseteq V (P)$ we put $X^\pm:=\{x^\pm:x\in X\}$

If $x \in P\setminus u$ is a neighbour of $u$ in $G$, then $P \cup \{(u, x)\}\setminus\{(x, x^-)\}$ (which is a path in $G$ with vertex
set $V (P)$) is said to have been obtained from $P$ by a rotation fixing $v$. A path obtained from $P$ by a (possibly empty) sequence of rotations fixing $v$ is a path derived from $P$. The set of starting vertices of paths derived from $P$, including $u$, will be denoted by $S(P)$. As all paths derived from $P$ have the same vertex set as $P$, we have $S(P) \subseteq V (P)$.

\begin{rem}
If some sequence of rotations can delete the edge $(x, x^-)$, call this edge a broken edge. Note that every interval of the original path not containing broken edges is traversed by all derived paths as a whole piece (however, the direction can change).
\end{rem}

\begin{defi}
For a graph $G$ and a subset $S \subseteq V (G)$, let $\partial S = \{v \in G\setminus S : \exists y \in  S, v\sim y\}$.
\end{defi}

\begin{lem}
Let $G$ be a graph, let $P = u \dots v$ be a longest path in $G$, and put $S := S(P)$. Then $\partial S \subseteq S^- \cup  S^+$.
\end{lem}

\begin{lem}
Let $G$ be a graph, let $P = u \dots v$ be a longest path in $G$, and put $S := S(P)$. If $deg(u) \ge 2$ then $ G$ has a cycle containing $S \cup \partial S$.
\end{lem}

\begin{cor}
Fix $k \ge 2$ and let $G$ be a graph such that for all $S \subseteq V (G)$ with $|S|\le k$, we have $|\partial S|\ge |2S|$. Then $G$ has a cycle of length at least $3k$.
\end{cor}

\subsection{Tournaments}
\begin{defi}
A tournament is a directed graph obtained by assigning a direction to every
edge of the complete graph. That is, it is an orientation of $K_n$.
\end{defi}

\begin{thm}
Every tournament has a Hamilton path.
\end{thm}

\begin{defi}
A tournament is strongly connected if for all $u, v$ there is a directed path from $u$ to $v$.
\end{defi}
\begin{thm}
A tournament $T$ is strongly connected if and only if it has a Hamilton cycle.
\end{thm}

\section{Kuratowski’s Theorem}

\begin{defi}
A subdivision of a graph $H$ is a graph obtained from $H$ by replacing the edges of $H$ by internally vertex disjoint paths of non-zero length with the same endpoints.
\end{defi}
\addtocounter{thm}{1}
\begin{rem}
If $G$ contains a subdivision of $H$, it also contains an $H$-minor.
\end{rem}

\begin{defi}
A Kuratowski graph is a graph which is a subdivision of $K_5$ or $K_{3,3}$. If $G$ is a graph and $H$ is a subgraph of $G$ which is a Kuratowski graph then we say that $H$ is a Kuratowski subgraph of $G$.
\end{defi}

\begin{thm}[Kuratowski 1930]
A graph is planar if and only if it has no Kuratowski subgraph.
\end{thm}

\begin{defi}
A straightline drawing of a planar graph G is a drawing in which every edge is a straight line.
\end{defi}

\begin{thm}
If $G$ is a graph with no Kuratowski subgraph then $G$ has a straightline drawing in the plane.
\end{thm}

\subsection{Convex drawings of 3-connected graphs}

\begin{defi}
A convex drawing of $G$ is a straightline drawing in which every non-outer face of $G$ is a convex polygon, and the outer face is the complement of a convex polygon. (That is, the boundary of each face is the boundary of a convex polygon).
\end{defi}

\begin{thm}[Tutte 1960]
If $G$ is a 3-connected graph which has no Kuratowski subgraphs then
$G$ has a convex drawing in the plane with no three vertices on a line.
\end{thm}

\begin{lem}[Thomassen 1980]
Every 3-connected graph $G$ with at least five vertices has an edge
$e$ such that $G\slash e$ is 3-connected.
\end{lem}

\begin{lem}
If $G$ has no Kuratowski subgraphs, then $G\slash e$ has no Kuratowski subgraph, for any edge $e\in E(G)$.
\end{lem}

\subsection{Reducing the general case to the 3-connected case}

\begin{defi}
Given a subdivision $H'$ of $H$, we call the vertices of the original graph branch vertices.
\end{defi}
\addtocounter{thm}{1}
\begin{fact}
We make three observations. 
\begin{enumerate}
\item  In a Kuratowski subgraph, there are three internally vertex-disjoint paths connecting any two
branch vertices. For $K_5$-subdivisions, we even have four such paths.
\item In a Kuratowski subgraph, there are four internally vertex-disjoint paths between any two pairs
of branch vertices.
\item Any cycle in a subdivision contains at least three branch vertices.
\end{enumerate}
\end{fact}

\begin{pro}
Let $G$ be a graph with at least 4 vertices which has no Kuratowski subgraph, and suppose that adding an edge-joining any pair of non-adjacent vertices creates a Kuratowski subgraph.
Then G is 3-connected.
\end{pro}

\section{Ramsey Theory}

\begin{pro}
Among six people it is possible to find three mutual acquaintances or three mutual non-acquaintances.
\end{pro}

As we shall see, given a natural number $s$, there is an integer $R$ such that if $n \ge R$ then every colouring of the edges of $K_n$ with red and blue contains either a red $K_s$ or a blue $K_s$. More generally, we define the Ramsey number $R(s, t)$ as the smallest value of $N$ for which every red-blue colouring of $K_N$ yields a red $K_s$ or a blue $K_t$. In particular, $R(s, t) = 1$ if there is no such $N$ such that in every red-blue
colouring of $K_N$ there is a red $K_s$ or a blue $K_t$. It is obvious that
$ R(s, t) = R(t, s)$ for every $s, t \ge 2$ and $R(s, 2) = R(2, s) = s$. 

\begin{thm}[Erdös, Szekeres]
The function $R(s, t)$ is finite for all $s, t \ge 2$. Quantitatively, if $s > 2 $ and $t > 2$ then $R(s, t)\le R(s - 1, t) + R(s, t - 1)$ and $R(s, t) \le \binom{s+t-2}{s-1}$.
\end{thm}

\begin{thm}
Given $k$ and $s_1, s_2, \dots , s_k$, if $N$ is sufficiently large, then every colouring of $K_N$ with $k$ colours is such that for some $i, 1 \le i \le k$, there is a $K_{s_i}$ coloured with the $i$-th colour. The
minimal value of $N$ for which this holds is usually denoted by $R_k(s_1, \dots , s_k)$ , and it satisfies $R_k(s_1, \dots , s_k)\le R_k-1(R(s_1, s_2), s_3, \dots , s_k)$.
\end{thm}

\begin{thm}
Let $min\{s, t\} > 3$. Then
$R^{(3)}(s, t) \le R(R^{(3)}(s - 1, t),R^{(3)}(s, t - 1)) + 1$
\end{thm}:

\subsection{Applications}

\begin{thm}[Erdos-Szekeres 1935]
Given an integer $m$, there exists a (least) integer $N(m)$ such
that every set of at least $N(m)$ points in the plane, with no three collinear, contains an $m$-subset forming a convex $m$-gon.
\end{thm}
\subsection{Bounds on Ramsey numbers}

\begin{thm}[Erdös 1947]
For $p \ge 3$, we have $R(p, p) > 2^{p/2}$.
\end{thm}

\begin{thm}
We have $R_k(3)\stackrel{def}{=}R_k(3,\dots,3)\le\left\lfloor e\cdot k!\right\rfloor+1$.
\end{thm}

\subsection{Ramsey theory for integers}

\begin{thm}[Schur 1916]
For every $k \ge 1$ there is an integer $m$ such that every $k$-colouring of [m] contains integers $x, y, z$ of the same colour such that $x + y = z.$
\end{thm}

\subsection{Graph Ramsey numbers}
\begin{defi}
Let $G_1,G_2$ be graphs. $R(G_1,G_2)$ is the minimal $N$ such that any red/blue colouring of $K_N$ contains either a red copy of $G_1$, or a blue copy of $G_2$.
\end{defi}

\begin{rem}
Note that $R(G1,G2) \le R(|G1||G2|)$.
\end{rem}

\begin{thm}[Chvatal 1977]
If $T$ is any $m$-vertex tree, then $R(T,K_n) = (m - 1)(n - 1) + 1$
\end{thm}

\section{Extremal problems}
\addtocounter{thm}{1}
\begin{defi}
$ex(n,H)$ is the maximal value of $e(G)$ among graphs $G$ with $n$ vertices containing no $H$ as a subgraph.
\end{defi}
\addtocounter{thm}{1}
\subsection{Turán’s theorem}
\begin{defi}
We call the graph $K_{n_1,\dots,n_r}$ with $|ni - nj|\le 1$ the Turán graph, denoted by $T_{n,r}$.
\end{defi}

\begin{thm}[Turan 1941]
Among all the $n$-vertex simple graphs with no $(r + 1)$-clique, $T_{n,r}$ is
the unique graph having the maximum number of edges.
\end{thm} 

\begin{que}
Let $a_1, \dots , a_n \in \mathbb{R}^d$ be vectors such that $|a_i|\ge 1$ for each $i in [n]$. What is the maximum number of pairs satisfying $|a_i + a_j| < 1$?
\end{que}

\begin{claim}
There are at most $\left\lfloor \frac{n^2}{4}\right\rfloor$ such pairs.
\end{claim}

\begin{defi}
For some fixed graph $H$, we define $\pi(H) = \lim\limits_{n\to\infty} ex(n,H)/\binom{n}{2}$
\end{defi}

\begin{thm}[Erdos-Stone]
Let $H$ be a graph of chromatic number $\chi(H) = r + 1$. Then for
every $\varepsilon > 0$ and large enough $n$, \[\left(1-\frac{1}{r}\right)\frac{n^2}{2}\le ex(n,H)\le\left(1-\frac{1}{r}\right)\frac{n^2}{2}+n^2\varepsilon\]
\end{thm}
 \addtocounter{thm}{1}
\subsection{Bipartite Turán Theorems}

\begin{thm}
If a graph $G$ on n vertices contains no 4-cycles, then $e(G)\le\left\lfloor\frac{n}{4}(1+\sqrt{4n-3})\right\rfloor$.
\end{thm}
\addtocounter{thm}{1}
\begin{thm}[Kovári-Sós-Turán]
For any integers $r \le s$, there is a constant $c$ such that every
$K_{r,s}$-free graph on $n$ vertices contains at most $cn^{1-\frac{1}{r}}$ edges. In other words, $ex(n,K_{r,s}) \le cn^{1-\frac{1}{r}}$ 
\end{thm}

\begin{thm}
There is $c$ depending on $k$ such that if $G$ is a graph on $n$ vertices that contains no copy of $C_{2k}$, then $G$ has at most $cn^{1+\frac{1}{k}}$ edges.
\end{thm}

\begin{que}
Given $n$ points in the plane, how many pairs can be at distance 1?
\end{que}

\begin{thm}[Erdos]
There are at most $cn^{3/2}$ pairs.
\end{thm}

\section{Exercises}
\subsection{Assignment 1}

\begin{problem}
Given a graph $G$ with vertex set $V = \{v_1,\dots,v_n\}$ we define the degree sequence of $G$ to be the list $d(v_1),\dots, d(v_n)$ of degrees in decreasing order.
\end{problem}

\begin{problem}

\end{problem}

\begin{problem}
Prove that if a graph $G$ is not connected then its complement $\overline{G}$ is connected. Is the converse also true? Nope.
\end{problem}

\begin{problem}
Show that every graph on at least two vertices contains two vertices of equal
degree.
\end{problem}

\begin{problem}
Prove that every graph with $n \ge 7$ vertices and at least $5n- 14$ edges contains a subgraph with minimum degree at least 6.
\end{problem}

\begin{problem}
Show that in a connected graph any two paths of maximum length share at least
one vertex.
\end{problem}

\begin{problem}
Prove that a graph is bipartite iff (if and only if) it contains no cycle of odd length.
\end{problem}

\subsection{Assignment 2}

\begin{problem}
Show that in a tree containing an even number of edges, there is at least one
vertex with even degree.
\end{problem}

\begin{problem}
Given a graph $G$ and a vertex $v \in V (G)$, $G - v$ denotes the subgraph of $G$ induced by the vertex set $V (G)\setminus\{v\}$. Show that every connected graph $G$ of order at least two contains vertices $x$ and $y$ such that both $G- x$ and $G-y$ are connected.
\end{problem}

\begin{problem}
Let $T$ be an $n$-vertex tree with exactly $2k$ odd-degree vertices. Prove that $T$ decomposes into $k$ paths (i.e. its edge-set is the disjoint union of $k$ paths).
\end{problem}

\begin{problem}
Prove that a connected graph $G$ is a tree if and only if any family of pairwise (vertex-)intersecting paths $P_1,\dots,P_k$ in $G$ have a common vertex.
\end{problem}

\begin{problem}
\begin{enumerate}[(a)]
\item Describe which Prüfer codes correspond to stars (i.e. to trees isomorphic to $K_{1,n-1}$).
\item Describe what trees correspond to Prüfer codes containing exactly 2 different values.
\end{enumerate}
\end{problem}

\begin{problem}
Let $T$ be a forest on vertex set $[n]$ with components $T_1,\dots,T_r$. Prove, by induction on $r$, that the number of spanning trees on $[n]$ containing $T$ is $n^{r-2}\prod_{i=1}^{r}|T_i|$. Deduce Cayley's formula.
\end{problem}

\subsection{Assignment 3}

\begin{problem}
Prove that a connected graph $G$ is $k$-edge-connected if and only if each block of $G$ is $k$-edge-connected.
\end{problem}

\begin{problem}
Let $G$ be a graph and suppose some two vertices $u, v \in V (G)$ are separated by $X \subseteq V (G)\setminus\{u,v\}$. Show that $X$ is a minimal separating set (i.e. there is no proper subset $Y$ ( $X$ that separates $u$ and $v$) if and only if every vertex in $X$ has a neighbor in the component
of $G-X$ containing $u$ and another in the component containing $v$.
\end{problem}

\begin{problem}
Show that if $G$ is a graph with $|V (G)| = n \ge k + 1$ and $\delta(G) \ge(n + k-2)/2$ then $G$ is $k$-connected.
\end{problem}

\begin{problem}
Prove that a graph $G$ with at least 3 vertices is 2-connected if and only if for any three vertices $x, y, z$ there is a path from $x$ to $z$ containing $y$.
\end{problem}

\begin{problem}
Let $G$ be a $k$-connected graph, where $k \ge 2$. Show that if $|V (G)| \ge 2k$ then $G$ contains a cycle of length at least $2k$.
\end{problem}

\subsection{Assignment 4}

\begin{problem}
Show that if $k > 0$ then the edge set of any connected graph with $2k$ vertices of odd degree can be split into $k$ trails.
\end{problem}

\begin{problem}
Let $G$ be a connected graph that has an Euler tour. Prove or disprove the
following statements.
\begin{enumerate}[(a)]
\item If $G$ is bipartite then it has an even number of edges.
\item If $G$ has an even number of vertices then it has an even number of edges.
\item For edges $e$ and $f$ sharing a vertex, $G$ has an Euler tour in which $e$ and $f$ appear consecutively.
\end{enumerate}
\end{problem}

\begin{problem}
Let $G$ be a connected graph on n vertices with minimum degree $\delta$. Show that 
\begin{enumerate}[(a)]
\item if $\delta \le \frac{n-1}{2}$ then $G$ contains a path of length $2\delta$, and
\item if $\delta \ge \frac{n-1}{2}$ then $G$ contains a Hamiltonian path.
\end{enumerate}
\end{problem}

\begin{problem}
Show that the maximum number of edges in a non-Hamiltonian graph on $n \ge 3$
vertices is $\binom{n-2}{1}+1$.
\end{problem}

\subsection{Assignment 5}

\begin{problem}
Let $G$ be a connected graph on more than 2 vertices such that every edge is
contained in some perfect matching of $G$. Show that $G$ is 2-edge-connected.
\end{problem}

\begin{problem}
\begin{enumerate}[(a)]
\item Let $G$ be a graph on $2n$ vertices that has exactly one perfect matching. Show that $G$ has at most $n^2$ edges.
\item Construct such a $G$ containing exactly $n^2$ edges for any $n \in N$.
\end{enumerate}
\end{problem}

\begin{problem}
Let $A$ be a finite set with subsets $A_1,\dots, A_n$, and let $d_1,\dots, d_n$ be positive integers. Show that there are disjoint subsets $D_k \subseteq A_k$ with $|D_k| = d_k$ for all $k \in [n]$ if and
only if $|\bigcup_{i\in I}A_i|\ge \sum_{i\in I}d_i$.
\end{problem}

\begin{problem}
Suppose $M$ is a matching in a bipartite graph $G = (A\cup B,E)$. We say that a path $P = a_1b_1\dots a_kb_k$ is an augmenting path in $G$ if $b_ia_{i+1}\in M$ for all $i\in [k - 1]$ and $a_1$ and $b_k$ are not covered by $M$. The name comes from the fact that the size of $M$ can be increased
by 
flipping the edges along $P$ (in other words, taking the symmetric difference of $M$ and $P$): by deleting the edges $b_ia_{i+1}$ from $M$ and adding the edges $a_ib_i$ instead.
\begin{enumerate}[(a)]
\item Prove Hall's theorem by showing that if Hall's condition is satisfied and M does not cover $A$, then there is an augmenting path in $G$.
\item Show that if $M$ is not a maximum matching (i.e. there is a larger matching in $G$) then the graph contains an augmenting path. Is this true for non-bipartite graphs as well?
\end{enumerate}
\end{problem}

\begin{problem}
Show that for $k\ge 1$, every $k$-regular ($k- 1$)-edge-connected graph on an even number of vertices contains a perfect matching.
\end{problem}

\subsection{Assignment 6}

\begin{problem}
Determine all positive integers $r$ and $s$, with $r \le s$, for which $K_{r,s}$ is planar.
\end{problem}

\begin{problem}
\begin{enumerate}[(a)]
\item Show that every planar graph has a vertex of degree at most 5. Is there a planar graph with minimum degree 5?
\item Show that any planar bipartite graph has vertex of degree at most 3. Is there a planar bipartite graph with minimum degree 3?
\end{enumerate}
\end{problem}

\begin{problem}
Show that a connected plane graph $G$ is bipartite iff all its faces have even length.
\end{problem}

\begin{problem}
Let $G$ be a graph on $n\ge 3$ vertices and $3n- 6 + k$ edges for some $k > 0$. Show that any drawing of $G$ in the plane contains at least $k$ crossing pairs of edges.
\end{problem}

\begin{problem}
Let $G$ be a plane graph with triangular faces and suppose the vertices are colored arbitrarily with three colors. Prove that there is an even number of faces that get all three colors.
\end{problem}

\begin{problem}
Let $S$ be a set of $n\ge 3$ points in the plane such that any two of them have distance at least 1. Show that there are at most $3n - 6$ pairs of distance exactly 1.
\end{problem}

\subsection{Assignment 7}

\begin{problem}
Are the following statements true?
\begin{enumerate}[(a)]
\item If $G$ and $H$ are graphs on the same vertex set, then $dg(G \cup H)\le dg(G) + dg(H)$.
\item If $G$ and $H$ are graphs on the same vertex set, then $\chi(G \cup H) \le \chi(G) + \chi(H)$.
\item Every graph $G$ has a $\chi(G)$-coloring where $\alpha(G)$ vertices get the same color.
\end{enumerate}
\end{problem}

\begin{problem}
$G$ has the property that any two odd cycles in it intersect (they share at least one vertex in common). Prove that $\chi(G) \le 5.$
\end{problem}

\begin{problem}
For a vertex $v$ in a connected graph $G$, let $G_r$ be the subgraph of $G$ induced by the vertices at distance $r$ from $v$. Show that $\chi(G) \le max_{0\le r\le n}\chi(G_r) + \chi(G_{r+1})$.
\end{problem}

\begin{problem}
Let $l$ be the length of the longest path in a graph $G$. Prove $\chi(G)\le l + 1$ using the fact that if a graph is not $d$-degenerate then it contains a subgraph of minimum degree at least $d + 1$.
\end{problem}

\begin{problem}
Suppose the complement of $G$ is bipartite. Show that $\chi(G) = \omega(G).$
\end{problem}

\subsection{Assignment 8}

\begin{problem}
For a given natural number $n$, let $G_n$ be the following graph with $\binom{n}{2}$ vertices and $\binom{n}{3}$ edges: the vertices are the pairs $(x, y)$ of integers with $1\le x < y \le n$, and for each triple $(x, y, z)$ with $1\le x < y < z\le n$, there is an edge joining vertex $(x, y)$ to vertex $(y, z)$. Show that for any natural number $k$, the graph $G_n$ is triangle-free and has chromatic number $\chi(G_n) > k$ provided $n > 2k$.
\end{problem}

\begin{problem}
Show that the theorem of Mader implies the following weakening of Hadwiger's
conjecture: Any graph $G$ with $\chi(G) \ge 2^{t-2} + 1$ has a $K_t$-minor.
\end{problem}

\begin{problem}
Find the edge-chromatic number of $K_n$ (don't use Vizing's theorem).
\end{problem}

\begin{problem}
Let $G$ be a connected $k$-regular bipartite graph with $k\ge 2$. Show, using König's theorem, that $G$ is 2-connected.
\end{problem}

\subsection{Assignment 9}

\begin{problem}
Prove that every graph G of maximum degree $\Delta$ has an equitable $(\Delta + 1)$-edge-coloring, i.e. one where each color class contains $\left\lfloor e=(\Delta + 1)\right\rfloor$ or $\left\lfloor e=(\Delta + 1)\right\rfloor$ edges, where $e$ is the number of edges in $G$.
\end{problem}

\begin{problem}
The cartesian product $H \times G $ of graphs $H$ and $G$ is the graph with vertex set $V (H)\times V (G)$, with an edge between $(v, u)$ and $(v', u')$ if $v = v'$ and $u$ is adjacent to $u'$ in $G$, or
if $u = u'$ and $v$ is adjacent to $v'$ in $H$. Prove that if $\chi'(H) = \Delta(H)$ then $\chi'(H\times G) = \Delta(H\times G)$
\end{problem}

\begin{problem}
Show that $\chi(C_n) = \chi_l(C_n)$ for any $n \ge 3$.
\end{problem}

\begin{problem}
Let $G$ be a bipartite graph on $n$ vertices. Prove that $\chi_l(G)\le 1 + \log_2(n)$ using the probabilistic method.
\end{problem}

\begin{problem}
Let $G$ be a complete $r$-partite graph with all parts of size 2. (In other words, $G$ is $K_{2r}$ minus a perfect matching.) Show, using a combination of induction and Hall's theorem, that $\chi_l(G) = r$.
\end{problem}

\subsection{Assignment 10}

\begin{problem}
How many spanning trees does $K_{r,s}$ have?
\end{problem}

\begin{problem}
Find the number of spanning trees of $K_n- e$ (the complete graph on n vertices with one edge removed) in two different ways:
\begin{enumerate}[(a)]
\item using the Matrix Tree Theorem, and 
\item using a double counting argument.
\end{enumerate} 
\end{problem}

\begin{problem}
In this exercise we prove the following alternative form of the matrix-tree theorem. For an $n$-vertex connected graph $G$, the number of spanning trees in $G$ is equal to the product of the  nonzero eigenvalues of the Laplacian matrix $M$ of $G$, divided by $n$. (This matrix $M$ is as in the lecture notes).
\end{problem}

\begin{problem}
\begin{enumerate}[(a)]
\item Prove that any $n$-by-$n$ bipartite graph with minimum degree $\delta > n/2$ contains a Hamilton cycle.
\item  Show that this is not necessarily the case if $\delta\le n/2$.
\end{enumerate}
\end{problem}

\subsection{Assignment 11}

\begin{problem}
For each of the following decide whether they are true (with justification) or false (by providing a counterexample).
\begin{enumerate}[(a)]
\item If every vertex of a tournament has positive in- and out-degree, then the tournament contains a directed Hamilton cycle. NOPE
\item If a tournament has a directed cycle, then it has a directed triangle. YES
\end{enumerate} 
\end{problem}

\begin{problem}
Let $G$ be a graph on $n \ge 3$ vertices with at least $\alpha(G)$ vertices of degree $n- 1$. Show that $G$ is Hamiltonian.
\end{problem}

\begin{problem}
Suppose $G$ is a graph on $n$ vertices where all the degrees are at least $\frac{n+q}{2}$. Show that any set $F$ of $q$ independent edges is contained in a Hamiltonian cycle.
\end{problem}

\subsection{Assignment 12}

\begin{problem}
The lower bound for $R(p, p)$ that you learn in the lectures is not a constructive proof: it merely shows the existence of a red-blue coloring not containing any monochromatic copy of $K_p$ by bounding the number of bad graphs. Give an explicit coloring on $K_{(p-1)^2}$ that proves $R(p, p) > (p - 1)^2$.
\end{problem}

\begin{problem}
Prove that for every fixed positive integer $r$, there is an $n$ such that any coloring of all the subsets of $[n] $ using $r$ colors contains two non-empty disjoint sets $X$ and $Y$ such that $X, Y $ and $X \cup Y$ have the same color.
\end{problem}

\begin{problem}
Prove that for every $k \ge 2$ there exists an integer $N$ such that every coloring of $[N]$ with $k$ colors contains three distinct numbers $a, b, c$ satisfying $ab = c$ that have the same color.
\end{problem}

\begin{problem}
Prove the following strengthening of Schur's theorem: for every $k\ge 2$ there is an $N$ such that any $k$-coloring of $[N]$ contains three distinct integers $a, b, c$ of the same color satisfying $a + b = c.$
\end{problem}

\begin{problem}
\begin{enumerate}[(a)]
\item Let $n\ge 1$ be an integer. Show that any sequence of $N\ge R(n, n)$  distinct numbers, $a_1,\dots, a_N$ contains a monotone (increasing or decreasing) subsequence of length $n$.
\item Let $k, l \ge1$ be integers and show that any sequence of $kl+1$ distinct numbers $a_1,\dots, a_{kl+1}$ contains a monotone increasing subsequence of length $k + 1$ or a monotone decreasing
subsequence of length $l + 1$.
\end{enumerate}
\end{problem}

\subsection{Assignment 13}

\begin{problem}
Let $H$ be an arbitrary fixed graph and prove that the sequence $ex(n,H)/\binom{n}{2}$ is (not necessarily strictly) monotone decreasing in $n$.
\end{problem}

\begin{problem}
Imitate the proof of Turan's theorem to show that among all the $n$-vertex $K_{r+1}$-free graphs, the Turan graph $T_{n,r}$ contains the maximum number of triangles (for any $r, n \ge 1$).
\end{problem}

\begin{problem}
Let $X$ be a set of $n$ points in the plane with no two points of distance greater than 1. Show that there are at most $\frac{n^2}{3}$ pairs of points in $X$ that have distance greater than $\frac{1}{\sqrt{2}}$.
\end{problem}


%\end{tiny}
\end{multicols}
\end{document}