%
%
%
%
%
%
%    SEARCH (Ctrl+F) for "addtocounter", to find all places, where elements were deleted
%
%	abbreviations used:
%	thm = theorem
%	vx = vertex
%	vxs = vertices
%	G = graph G
%	gr. = graph
%	# = number of
%	conn. = connected
%	iff = if and only if





\documentclass[11pt, fleqn, a4paper, landscape]{article}
\usepackage{listings}
\usepackage{color}
\usepackage{datetime}
\usepackage[margin=2cm]{geometry}
\usepackage[utf8]{inputenc}
\usepackage{textcomp}
\usepackage{amssymb, etoolbox}
\usepackage{booktabs, longtable}
\usepackage{amsmath, amsthm}
\usepackage{mathtools}

\setlength{\mathindent}{0pt} 
\setlength{\arraycolsep}{2pt} %reducing whitespace in matrices in small fonts

\usepackage{pgfplots}
	\pgfplotsset{
		compat=1.12
	}
\allowdisplaybreaks
\usepackage[compact]{titlesec} 
\titlespacing{\section}{0pt}{0pt}{0pt} % this and the following, are for spaces around the mentioned parts
\titlespacing{\subsection}{0pt}{0pt}{0pt}
\titlespacing{\subsubsection}{0pt}{0pt}{0pt}
\titlespacing{\paragraph}{0pt}{0pt}{0pt}
%\newcommand{\sectionbreak}{\clearpage} %start a new section on a new page

\usepackage{multicol} % multicolumn layout

\setlength\parindent{0pt} %Do not indent after an empty line


%\renewcommand{\baselinestretch}{1.5} %Zeilenabstand von 1.5
\usepackage[shortlabels]{enumitem} % das und folgendes für Abstand zwischen items [shortlabels um i),ii) etc. zu erlauben]
\setlist[enumerate]{nolistsep}
\setlist[itemize]{nolistsep} % depends on [shortlabels]{enumitem}
	%indentation of enumerate and itemize, numebrs indicate the level of nestedness, depends on usepackage{enumitem}
	\setlist[enumerate,1]{leftmargin=.4cm}
	\setlist[enumerate,2]{leftmargin=.4cm}
	\setlist[enumerate,3]{leftmargin=.6cm}
	\setlist[enumerate,4]{leftmargin=.6cm}
	\setlist[itemize,1]{leftmargin=.4cm}
	\setlist[itemize,2]{leftmargin=.4cm}
	\setlist[itemize,3]{leftmargin=.6cm}
	\setlist[itemize,4]{leftmargin=.6cm}

\makeatletter
\@addtoreset{problem}{subsection}
\makeatother

\newtheoremstyle{plain}% name of the style to be used
  {}   % ABOVESPACE
  {\parskip}   % BELOWSPACE
  {\itshape}  % BODYFONT
  {0pt}       % INDENT (empty value is the same as 0pt)
  {\bfseries} % HEADFONT
  {.}         % punctuation between head and body
  {0pt} % space after theorem head; " " = normal interword space
  {}          % CUSTOM-HEAD-SPEC
  
\newtheoremstyle{definition}
  {}% measure of space to leave above the theorem. E.g.: 3pt
  {\parskip}% measure of space to leave below the theorem. E.g.: 3pt
  {}% name of font to use in the body of the theorem
  {0pt}% measure of space to indent
  {\bfseries}% name of head font
  {.}% punctuation between head and body
  {0pt}% space after theorem head; " " = normal interword space
  {}% Manually specify head

\theoremstyle{plain} % default, italic, extra space
\newtheorem{thm}{Thm}
\newtheorem{lem}[thm]{Lem}
\newtheorem{pro}[thm]{Prop}
\newtheorem{cor}[thm]{Cor}

\theoremstyle{remark} % roman, no space
\newtheorem{rem}[thm]{Rem}
\newtheorem{nota}[thm]{Not}
\newtheorem{claim}[thm]{Claim}
\newtheorem{fact}{Fact}
\newtheorem{que}[thm]{Question}
\newtheorem{problem}{P}

\theoremstyle{definition} % roman, extra space
\newtheorem{defi}[thm]{Def}


% Modify theorem counter to match that of the section unit
\preto{\section}{\renewcommand{\thethm}{\thesection.\arabic{thm}}}%


% Reset the counter at every sectional unit
\makeatletter
\@addtoreset{thm}{section}
\makeatother

\begin{document}

\setlength{\abovedisplayskip}{0pt}
\setlength{\belowdisplayskip}{0pt}
\setlength{\abovedisplayshortskip}{0pt}
\setlength{\belowdisplayshortskip}{0pt}

\begin{multicols}{5}
%\begin{tiny}

\section{Basic notions}
\addtocounter{subsection}{1} 
%\subsection{Graphs}
%\begin{defi}
%A $G$ is a pair $G = (V,E)$ where $V$ is a set of vxs and $E$ is a (multi)set of unordered pairs of vxs. The elements of $E$ are called edges. We write $V (G)$ for the set of vxs and $E(G)$ for the set of edges of a $G$. Also, $|G| = |V (G)|$ denotes the \#
%vxs and $e(G) = |E(G)|$ denotes the \# edges.
%\end{defi}
\addtocounter{thm}{1}
%\begin{defi}
%A loop is an edge $(v, v)$ for some $v\in V$ . An edge $e = (u, v)$ is a multiple edge if it appears multiple times in $E$. A gr. is simple if it has no loops or multiple edges.
%\end{defi}
\addtocounter{thm}{1}
%\begin{defi}
%\begin{itemize}
%\item Vertices $u, v$ are adjacent in $G$ if $(u, v)\in E(G)$.
%\item An edge $e \in E(G)$ is incident to a vx $v \in V (G)$ if $v \in e$.
%\item Edges $e, e'$ are incident if $e\cap e'\ne \emptyset$.
%\item If $(u, v) \in E $ then $v$ is a neighbour of $u$.
%\end{itemize}
%\end{defi}
\addtocounter{thm}{1}
\addtocounter{thm}{1}

\subsection{Graph isomorphism}
\addtocounter{thm}{2}
%\begin{defi}
% Let $G_1 = (V_1,E_1)$ and $G_2 = (V_2,E_2)$ be graphs. An isomorphism $\phi: G_1 \to G_2$ is a bijection (a one-to-one correspondence) from $V_1$ to $V_2$ such that $(u, v) \in E_1$ iff $(\phi(u), \phi(v))\in E_2$. We say $G_1$ is isomorphic to $G_2$ if there is an isomorphism between them.
%\end{defi}
\addtocounter{thm}{1}
\begin{rem}
Isomorphism is an equivalence relation of graphs. (reflexive, symmetric, transitive)
%\begin{itemize}
%\item Any gr. is isomorphic to itself
%\item If $G_1$ is isomorphic to $G_2$ then $G_2$ is isomorphic to $G_1$
%\item If $G_1 $ is isomorphic to $G_2$ and $G_2$ is isomorphic to $G_3$, then $G_1$ is isomorphic to $G_3$.
%\end{itemize}
\end{rem}

\addtocounter{thm}{1}
%\begin{defi}
%An unlabelled gr. is an isomorphism class of graphs.
%\end{defi} 

\subsection{The adjacency and incidence matrices}

%Let $[n] = \{1, \dots , n\}.$

%\begin{defi}
%Let $G = (V,E)$ be a gr. with $V = [n].$ The adjacency matrix $A = A(G)$ is the gr. with $ V = [n]$. The adjacency matrix $A = A(G)$ is the
%$n \times n $ symmetric matrix defined by
%\[a_{ij}=\begin{cases}1 & if (i,j)\in E\\ 0 & otherwise\end{cases}\]
%\end{defi} 
\addtocounter{thm}{2}
%\begin{rem}
%Any adjacency matrix $A $ is real and symmetric, hence the spectral theorem proves that $A$ has an orthogonal basis of eigenvalues with real eigenvectors. This important fact allows us to use spectral methods in gr. theory. Indeed, there is a large subfield of gr. theory called spectral gr. theory.
%\end{rem}
\addtocounter{thm}{1}

%\begin{defi}
%Let $G = (V,E)$ be a gr. with $V = \{v_1, \dots , v_n\}$ and $E = \{e_1, \dots , e_m\}$. Then the incidence matrix $B = B(G)$ of $G$ is the $n\times m$ matrix defined by \[b_{ij}=\begin{cases}1 & if v_i\in e_j\\ 0 & otherwise\end{cases}\]
%\end{defi}
\addtocounter{thm}{3}
%\begin{rem}
%Every column of $B$ has $|e| = 2$ entries 1.
%\end{rem}

\subsection{Degree}

%\begin{defi}
%Given $G = (V,E)$ and a vx $v \in V$ , we define the neighbourhood $N(v)$ of $v$ to
%be the set of neighbours of $v$. Let the degree $d(v)$ of $v$ be $|N(v)|$, the \# neighbours of $v$. A vx $v$ is isolated if $d(v) = 0.$
%\end{defi}
\addtocounter{thm}{1}
%\begin{rem}
%$d(v)$ is the \# 1s in the row corresponding to v in the adjacency matrix $A(G)$ or the incidence matrix $B(G)$.
%\end{rem}
\addtocounter{thm}{1} 
\addtocounter{thm}{1}
\begin{fact}
For any $G$ on the vx set $[n]$ with adjacency and incidence matrices $A$ and $B$, we have $BB^T = D + A$, where
$D=
\begin{pmatrix}
d(1) & 0 & 0  \\
0 & \ddots & 0  \\
0 & 0 & d(n)  
\end{pmatrix}
$  
\end{fact}

%\begin{nota}
%The minimum degree of a $G$ is denoted $\delta(G)$, the maximum degree is denoted $\Delta(G)$. The average degree is \[\overline{d}(G) =\frac{\sum_{v\in G}}{|V(G)|}\] Note that $\delta\le\overline{d}\le\Delta$.
%\end{nota}
\addtocounter{thm}{1}

%\begin{defi}
%A $G$ is $d$-regular iff all vxs have degree $d$.
%\end{defi}
\addtocounter{thm}{2}
\begin{pro}
For every $G = (V,E)$, $\sum_{v\in G}d(G)=2|E|$
\end{pro}

\begin{cor}
Every gr. has an even \# vxs of odd degree.
\end{cor}

\subsection{Subgraphs}

%\begin{defi}
%A gr. $H = (U, F)$ is a subgr. of a gr. $G = (V,E)$ if $U \subseteq V$ and $F \subseteq E$. If $U = V$ then $H$ is called spanning.
%\end{defi}

%\begin{defi}
%Given $G = (V,E)$ and $U\subseteq  V (U \ne \emptyset)$, let $G[U]$ denote the gr. with vx set $U$ and edge set $E(G[U]) = \{e \in E(G) : e \subseteq U\}$. (We include all the edges of $G$ which have both
%endpoints in $U$). Then $G[U]$ is called the subgr. of $G$ induced by $U$.
%\end{defi}
\addtocounter{thm}{3}
\addtocounter{subsection}{1} 
%\subsection{Special graphs}

%\begin{itemize}
%\item  $K_n$ is the complete graph, or a clique. Take $n$ vxs and all possible edges connecting them.
%\item An empty gr. has no edges.
%\item $G = (V,E)$ is bipartite if there is a partition $V = V_1 \cup V_2$ into two disjoint sets such that each $e \in E(G)$ intersects both $V_1$ and $V_2$.
%\item $K_{n,m}$ is the complete bipartite graph. Take $n + m$ vxs partitioned into a set $ A$ of size $n$ and a set $B$ of size $m$, and include every possible edge between $A$ and $B$.
%\end{itemize}
\addtocounter{thm}{1}
\addtocounter{thm}{1}
\subsection{Walks, paths and cycles}
%
%\begin{defi}
%A walk in $G$ is a sequence of vxs $v_0, v_1, \dots , v_k$, and a sequence of edges $(v_i, v_{i+1}) \in E(G)$. A walk is a path if all $v_i$ are distinct. If for such a path with $k \ge 2$, $(v_0, v_k)$ is also
%an edge in $G$, then $v_0, v_1 \dots , v_k, v_0$ is a cycle. For multigraphs, we also consider loops and pairs of multiple edges to be cycles.
%\end{defi}
%
%\begin{defi}
%The length of a path, cycle or walk is the \# edges in it.
%\end{defi}
\addtocounter{thm}{3}
\begin{pro}
Every walk from u to v in G contains a path between u and v.
\end{pro}

\begin{pro}
Every $G$ with minimum degree $ \delta\ge 2$ contains a path of length $\delta$ and a cycle of length at least $\delta + 1$.
\end{pro}

\begin{rem}
Note that we have also proved that a gr. with minimum degree $\delta\ge 2$ contains cycles of at least $\delta-1$ different lengths. This fact, and the statement of Proposition 1.32, are both tight, to see this, consider the complete gr. $G = K_{\delta+1}$.
\end{rem} 

\subsection{Connectivity}

%\begin{defi}
%A $G$ is conn. if for all pairs $u, v \in G$, there is a path in $G$ from $u$ to $v$.
%\end{defi}
%Note that it suffices for there to be a walk from u to v, by Proposition 1.31.
\addtocounter{thm}{2}
%\begin{defi}
%A (conn.) component of $G$ is a conn. subgr. that is maximal by inclusion. We say $G$ is conn. iff it has one conn. component.
%\end{defi}
\addtocounter{thm}{2}
\begin{pro}
A gr. with $n$ vxs and $m$ edges has at least $n-m$ conn. components.
\end{pro}

\subsection{Graph operations and parameters}

%\begin{defi}
%Given $G = (V,E)$, the complement $\overline{G}$ of $G$ has the same vx set $V$ and $(u, v) \in E(\overline{G})$ iff $(u, v) \notin E(G)$.
%\end{defi}
\addtocounter{thm}{2}
%\begin{defi}
%A clique in $G$ is a complete subgr. in $G$. An independent set is an empty
%induced subgr. in $G$.
%\end{defi}
\addtocounter{thm}{2}
\begin{nota}
Let $\omega(G)$ denote the \# vxs in a maximum-size clique in $G$, let $\alpha(G)$ denote the \# vxs in a maximum-size independent set in $G$.
\end{nota}

\begin{claim}
A vx set $U\subseteq V (G)$ is a clique iff $U\subseteq V(\overline{G})$ is an independent set.
\end{claim}

\begin{cor}
We have $\omega(G)=\alpha(\overline{G})$ and $\alpha(G)=\omega(\overline{G})$.
\end{cor}

\section{Trees}
\subsection{Trees}
%\begin{defi}
%A gr. having no cycle is acyclic. A forest is an acyclic graph, a tree is a conn. acyclic graph. A leaf is a vx of degree 1.
%\end{defi}
\addtocounter{thm}{2}
\begin{lem}
Every finite tree with at least two vxs has at least two leaves. Deleting a leaf from an $n$-vx tree produces a tree with $n-1$ vxs.
\end{lem}

\subsection{Equivalent definitions of trees}

\begin{thm}
For an $n$-vx simple $G$ (with $n \ge 1$), the following are equivalent (and
characterize the trees with $n$ vxs).
(a) $G$ is conn. and has no cycles.
(b) $G$ is conn. and has $n - 1$ edges.
(c) $G$ has $n - 1$ edges and no cycles.
(d) For every pair $u, v \in V (G)$, there is exactly one $u, v$-path in $G$.
\end{thm}

%\begin{defi}
%An edge of a gr. is a cut-edge if its deletion disconnects the graph.
%\end{defi}
\addtocounter{thm}{1}
\begin{lem}
An edge contained in a cycle is not a cut-edge.
\end{lem}


%\begin{defi}
%Given a conn. $G$, a spanning tree $T$ is a subgr. of $G$ which is a tree and contains every vx of $G$.
%\end{defi}
\addtocounter{thm}{1}

\begin{cor}
\begin{itemize}
\item  Every conn. gr. on n vxs has at least n - 1 edges and contains a spanning tree,
\item Every edge of a tree is a cut-edge,
\item Adding an edge to a tree creates exactly one cycle.
\end{itemize}
\end{cor}

\subsection{Cayley’s formula}
\addtocounter{thm}{1}\addtocounter{thm}{1}
\begin{thm}[Cayley’s Formula]
There are $n^{n-2}$ trees with vx set $[n]$.
\end{thm}

%\begin{defi}[Prüfer code]
%Let $T$ be a tree on an ordered set $S$ of n vxs. To compute the
%Prüfer sequence $f(T)$, iteratively delete the leaf with the smallest label and append the label of its neighbour to the sequence. After $n - 2$ iterations a single edge remains and we have produced a sequence $f(T)$ of length $n - 2$.
%\end{defi}
\addtocounter{thm}{2}
\begin{pro}
For an ordered $n$-element set $S$, the Prüfer code $f$ is a bijection between the trees with vx set $S$ and the sequences in $S^{n-2}$.
\end{pro}
\addtocounter{thm}{1}
%\begin{defi}
%A directed graph, or digr. for short, is a vx set and an edge (multi-)set of ordered pairs of vxs. Equivalently, a digr. is a (possibly not-simple) gr. where each edge is assigned a direction. The out-degree (respectively in-degree) of a vx is the \# edges
%incident to that vx which point away from it (respectively, towards it).
%\end{defi}
\addtocounter{thm}{1}

\section{Connectivity}
\subsection{Vertex connectivity}

%\begin{defi}
%A vx cut in a conn. gr. $G = (V,E)$ is a set $ S \subseteq V$ such that $G\setminus S := G[V \setminus S]$ has more than one conn. component. A cut vx is a vx $v$ such that $\{v\}$ is a cut.
%\end{defi}
%\begin{defi}
%$G$ is called $k$-conn. if $|V (G)|> k$ and if $G\setminus X$ is conn. for every set $X \subseteq V$ with $|X|< k$. In other words, no two vxs of $G$ are separated by fewer than $k$ other vxs. Every (non-empty) gr. is 0-conn. and the 1-conn. graphs are precisely the non-trivial conn. graphs. The greatest integer $k$ such that $G$ is k-conn. is the connectivity $\kappa (G)$ of $G$.
%17
%\item $G = K_n: \kappa (G) = n - 1$
%\item $G = K_{m,n}, m \le n: \kappa (G) = m$. Indeed, let $G$ have bipartition $A \cup B$, with $|A|= m$  and $|B|= n$. Deleting $A$ disconnects the graph. On the other hand, deleting $S \subset V$ with $|S|< m$ leaves both $A\setminus S$ and $B\setminus S$ non-empty and any $a \in A\setminus S$ is conn. to any $b \in B \setminus S$. Hence $G\setminus S$ is conn.
%\end{defi}
\addtocounter{thm}{2}
\begin{pro}
For every $G$, $\kappa (G) \le \delta(G)$.
\end{pro} 

\begin{rem}
High minimum degree does not imply connectivity. Consider two disjoint copies of $K_n$.
\end{rem}
\begin{thm}[Mader 1972]
Every gr. of average degree at least $4k$ has a $k$-conn. subgraph.
\end{thm}

\subsection{Edge connectivity}
%\begin{defi}
%A disconnecting set of edges is a set $F \subseteq E(G)$ such that $G\setminus F$ has more than one component. Given $S$, $T \subset V (G)$, the notation $[S, T]$ specifies the set of edges having one endpoint
%in $S$ and the other in $T$. An edge cut is an edge set of the form $[S, S]$, where $S$ is a non-empty proper subset of $V (G)$. A gr. is $k$-edge-conn. if every disconnecting set has at least $k$ edges.
%The edge-connectivity of $G$, written $\kappa'(G)$, is the minimum size of a disconnecting set. One edge disconnecting G is called a bridge.
%
%$G = Kn: \kappa'(G) = n - 1.$
%\end{defi}
\addtocounter{thm}{2}
\begin{rem}
An edge cut is a disconnecting set but not the other way around. However, every minimal disconnecting set is a cut.
\end{rem}

\begin{thm}
$\kappa (G) \le \kappa'(G) \le \delta(G).$
\end{thm}

\subsection{Blocks}
%\begin{defi}
%A block of a $G$ is a maximal conn. subgr. of $G$ that has no cut-vx.
%If $G$ itself is conn. and has no cut-vx, then $G$ is a block.
%\end{defi}
\addtocounter{thm}{2}
\begin{rem}
If a block $B$ has at least three vxs, then $B$ is 2-conn. If an edge is a block of $G$ then it is a cut-edge of $G$.
\end{rem}

\begin{pro}
Two blocks in a gr. share at most one vx.
\end{pro}

%\begin{defi}
%The block gr. of a $G$ is a bipartite gr. $H$ in which one partite set consists of the cut-vxs of $G$, and the other has a vx $b_i$ for each block $B_i$ of $G$. We include $(v, b_i)$ as an edge of $H$ iff $v \in B_i.$
%\end{defi}
\addtocounter{thm}{2}
\begin{pro}
The block gr. of a conn. gr. is a tree.
\end{pro}

\subsection{2-conn. graphs}

%\begin{defi}
%Two paths are internally disjoint if neither contains a non-endpoint vx of the other. We denote the length of the shortest path from $u$ to $v$ (the distance from $u$ to $v$) by $d(u, v)$.
%\end{defi}
\addtocounter{thm}{1}
\begin{thm}[Whitney 1932]
A $G$ having at least three vxs is 2-conn. if and only
if each pair $u, v \in V (G)$ is conn. by a pair of internally disjoint $u, v$-paths in $G$.
\end{thm}

\begin{cor}
$G$ is 2-conn. and $|G|\ge 3$ iff every two vxs in $G$ lie on a common cycle.
\end{cor}
\subsection{Menger’s Thm}

%\begin{defi}
%Let $A,B \subseteq V$ . An $A-B$ path is a path with one endpoint in $A$, the other endpoint in $B$, and all interior vxs outside of $A \cup B$. Any vx in $A - B$ is a trivial $A-B$ path.
%
%If $X \subseteq V$ (or $X \subseteq E$) is such that every $A-B$ path in $G$ contains a vx (or an edge) from $X$, we say that $X$ separates the sets $A$ and $B$ in $G$. This implies in particular that $A \cap B \subseteq X$.
%\end{defi}
\addtocounter{thm}{1}
\begin{thm}[Menger 1927]
Let $G = (V,E)$ be a gr. and let $S, T \subseteq V$ . Then the maximum
\# vx-disjoint $S-T$ paths is equal to the minimum size of an $S-T$ separating vx set.
\end{thm}

\begin{cor}
For $S \subseteq V$ and $v\in V \setminus S$, the minimum \# vxs distinct from $v$ separating $v$ from $S$ in $G$ is equal to the maximum \# paths forming an $v-S$ fan in $G$. (that is, the maximum \# $\{v\}-S$ paths which are disjoint except at $v$).
\end{cor}

%\begin{defi}
%The line gr. of $G$, written $L(G)$, is the gr. whose vxs are the edges of $G$, with $(e, f) \in E(L(G))$ when $e = (u, v)$ and $f = (v,w)$ in $G$ (i.e. when $e$ and $f$ share a vx).
%\end{defi}
\addtocounter{thm}{2}
\begin{cor}
Let u and v be two distinct vxs of G.
\begin{enumerate}
\item If $(u, v) \notin E$, then the minimum \# vxs different from $u, v$ separating $u$ from $v$ in $G$ is equal to the maximum \# internally vx-disjoint $u-v$ paths in $G$.
\item The minimum \# edges separating $u$ from v in G is equal to the maximum \# edge-disjoint $u-v$ paths in $G$.
\end{enumerate}
\end{cor}

\begin{thm}
(Global Version of Menger’s Theorem)
\begin{enumerate}
\item A gr. is k-conn. iff it contains k internally vx-disjoint paths between any
two vxs.
\item A gr. is k-edge-conn. iff it contains k edge-disjoint paths between any two
vxs.
\end{enumerate}
\end{thm}

\section{Eu. \& Ha. cyc.}
\subsection{Eul. trails \& tours}
\addtocounter{thm}{2}
%\begin{defi}
%A trail is a walk with no repeated edges.
%\end{defi}

%\begin{defi}
%An Eulerian trail in a (multi)gr. $G = (V,E)$ is a walk in $G$ passing through every edge exactly once. If this walk is closed (starts and ends at the same vx) it is called an Eulerian tour.
%\end{defi}
\addtocounter{thm}{2}
\begin{thm}
A conn. (multi)gr. has an Eulerian tour iff each vx has even degree.
\end{thm}

\begin{lem}
Every maximal trail in an even gr. (i.e., a gr. where all the vxs have even degree) is a closed trail.
\end{lem} 

\begin{cor}
A conn. multi$G$ has an Eulerian trail iff it has either 0 or 2 vxs of odd degree.
\end{cor}

\subsection{Hamilton paths and cycles}
%\begin{defi}
%A Hamilton path/cycle in a $G$ is a path/cycle visiting every vx of $G$ exactly once. A $G$ is called Hamiltonian if it contains a Hamilton cycle.
%\end{defi}
\addtocounter{thm}{2}
\begin{pro}
If $G$ is Hamiltonian then for any set $S \subseteq V$ the gr. $G\setminus S$ has at most $|S|$ conn. components.
\end{pro}

\begin{cor}
If a conn. bipartite gr. $G = (V,E)$ with bipartition $V = A\cup B$ is Hamiltonian then $|A|=|B|$.
\end{cor}
\addtocounter{thm}{1}
\begin{thm}[Dirac 1952]
If $G$ is a simple gr. with $n \ge 3$ vxs and if $\delta(G) \ge n/2$, then $G$ is Hamiltonian.
\end{thm}
\addtocounter{thm}{1}
\begin{thm}[Ore 1960]
If $G$ is a simple gr. with $n\ge 3$ vxs such that for every pair of
non-adjacent vxs $u, v$ of $G$ we have $d(u) + d(v)\ge |G|$, then $G $ is Hamiltonian.
\end{thm}
\section{Matchings}
%\begin{defi}
%A set of edges $M \subseteq E(G)$ in a $G$ is called a matching if $e \cap e' = \emptyset$ for any pair of edges $e, e' \in M$.
%
%A matching is perfect if $|M|=\frac{|V (G)|}{2}$ , i.e. it covers all vxs of $G$. We denote the size of the maximum matching in $G$, by $\nu(G)$.
%
%$G=K_n; \nu(G)=\left\lfloor\frac{n}{2}\right\rfloor$
%
%$G=K_{s,t}; s\le t, \nu(G)=s$
%
%$\nu(Petersen Graph)=5$
%\end{defi}
\addtocounter{thm}{2}
\begin{rem}
A matching in a $G$ corresponds to an independent set in the line gr. $L(G)$.
\end{rem}
%\begin{defi}
%A set of vxs $T \subseteq V (G)$ of a gr.  $G$ is called a cover of $G$ if every edge $e \in E(G)$ intersects $T (e \cap T \neq \emptyset)$, i.e., $G \setminus T$ is an empty graph. Then, $\tau(G)$ denotes the size of
%the minimum cover.
%
%$G = Kn: \tau(G) = n - 1$
%
%$G = K_{s,t}, s \le t: \tau(G) = s$
%
%$\tau(Petersen Graph)=6$
%\end{defi}
\addtocounter{thm}{2}
\begin{pro}
$\nu(G) \le\tau(G) \le 2\nu(G).$
\end{pro}
%\subsection{Real-world applications of matchings}
\addtocounter{subsection}{1}
\subsection{Hall’s Theorem}
\begin{thm}[Hall 1935]
A bipartite gr. $G = (V,E)$ with bipartition $V = A\cup B$ has a matching
covering $A$ iff $|N(S)|\ge|S|\forall S \subseteq A$
\end{thm}

\begin{cor}
 If in a bipartite gr. $G = (A\cup B,E)$ we have $|N(S)|\ge|S|-d$ for every set $S \subseteq A$ and some fixed $d\in\mathbb{N}$, then $G$ contains a matching of cardinality $|A|- d$.
\end{cor}

\begin{cor}
If a bipartite gr. $G = (A \cup B,E) $ is $k$-regular with $k \ge 1$, then $G$ has a perfect matching.
\end{cor}

\begin{cor}
Every regular gr. of positive even degree has a 2-factor (a spanning 2-regular subgraph).
\end{cor}

\begin{rem}
A 2-factor is a disjoint union of cycles covering all the vxs of a graph
\end{rem}

%\begin{defi}
%Let $A_1, \dots ,A_n$ be a collection of sets. A family $\{a_1, \dots , a_n\}$  is called a system of distinct representatives (SDR) if all the $a_i$ are distinct, and $a_i \in A_i$ for all $i$.
%\end{defi}
\addtocounter{thm}{1}
\begin{cor}
A collection $A_1, \dots ,A_n$ has an SDR iff for all $I \subseteq [n]$ we have $|\bigcup_{i\in I} A_i|\ge|I|$.
\end{cor}
\addtocounter{thm}{1}
\begin{thm}[König 1931]
If $G = (A \cup B,E)$ is a bipartite graph, then the maximum size of a
matching in $G$ equals the minimum size of a vx cover of $G$.
\end{thm}

\subsection{Matchings in general graphs: Tutte’s Theorem}
Given a $G$, let $q(G)$ denote the \# its odd components, i.e. the ones of odd order. If G has a perfect matching then clearly
$q(G\setminus S) \le|S| \forall S \subseteq V (G)$
since every odd component of $G\setminus S$ will send an edge of the matching to $S$, and each such edge covers a different vx in $S$.

\begin{thm}[Tutte 1947]
A $G$ has a perfect matching iff $q(G\setminus S) \le|S|$ for
all $S \subseteq V (G)$.
\end{thm}

\begin{cor}[Petersen 1891]
Every 3-regular gr. with no cut-edge has a perfect matching.
\end{cor}
\addtocounter{thm}{1}
\begin{cor}[Berge 1958]
The largest matching in an $n$-vx $G$ covers $n+min_{S\subseteq V (G)}(|S|- q(G\setminus S))$ vxs.
\end{cor}

\section{Planar Graphs}
%\begin{defi}
%A polygonal path or polygonal curve in the plane is the union of many line segments such that each segment starts at the end of the previous one and no point appears in more than one segment except for common endpoints of consecutive segments. In a polygonal $u, v$-path, the beginning of the first segment is $u$ and the end of the last segment is $v$.
%
%A drawing of a $G$ is a function that maps each vx $v\in V (G)$ to a point $f(v)$ in the plane and each edge $uv$ to a polygonal $f(u), f(v)$-path in the plane. The images of vxs are distinct.
%A point in $f(e)\cup f(e')$ other than a common end is a crossing. A gr. is planar if it has a drawing without crossings. Such a drawing is a planar embedding of $G$. A plane gr. is a particular drawing of a planar gr. in the plane with no crossings.
%\end{defi}
\addtocounter{thm}{2}
\addtocounter{thm}{1}
%\begin{rem}
%We get the same class of graphs if we only require images of edges to be continuous curves. This is because any continuous line can be arbitrarily accurately approximated by a polygonal curve.
%\end{rem}

%\begin{defi}
%An open set in the plane is a set $U \subset \mathbb{R}^2$ such that for every $p\in U$, all points within some small distance from $p$ belong to $U$. A region is an open set $U$ that contains a polygonal $u, v$-path for every pair $u, v \in U$ (that is, it is “path-conn.”). The faces of a plane gr. are the maximal regions of the plane that are disjoint from the drawing.
%\end{defi}
\addtocounter{thm}{1}
\begin{thm}[Jordan curve theorem]
A simple closed polygonal curve C consisting of finitely
many segments partitions the plane into exactly two faces, each having C as boundary.
\end{thm}

\begin{rem}
This is not true in three dimensions. In $\mathbb{R}$ there is a surface called the Möbius band which has only one side.
\end{rem}

\begin{rem}
The faces of $G$ are pairwise disjoint (they are separated by the edges of G). Two points are in the same face iff there is a polygonal path between them which does not cross an edge of $G$. Also, note that a finite gr. has a single unbounded face (the area “outside” of the graph).
\end{rem}

\begin{pro}
A plane forest has exactly one face.
\end{pro}

%\begin{defi}
%The length of the face $f$ in a planar embedding of $G$ is the sum of the lengths of the walks in $G$ that bound it.
%\end{defi}
\addtocounter{thm}{2}
\begin{pro}
If $l(f_i)$ denotes the length of a face $f_i$ in a plane $G$, then $2e(G) = \sum l(f_i)$.
\end{pro}

\begin{thm}[Euler's formula 1758]
If a conn. plane $G$ has exactly $n$ vxs, $e$ edges
and $f$ faces, then $n - e + f = 2$.
\end{thm}

%\begin{rem}
%The fact that deleting an edge in a cycle decreases the \# faces by one can be proved formally using the Jordan curve theorem.
%\end{rem}
\addtocounter{thm}{1}
\begin{thm}
If $G$ is a planar gr. with at least three vxs, then $e(G) \le 3|G|- 6$. If $G$ is also triangle-free, then $e(G)\le 2|G|- 4$.
\end{thm}

\begin{cor}
If $G$ is a planar bipartite $n$-vx gr. with $n \ge 3$ vxs then $G$ has at most $2n - 4$ edges.
\end{cor}

\begin{cor}
$K_5$ and $K_{3,3}$ are not planar.
\end{cor}

\begin{rem}[Maximal planar graphs / triangulations]
The proof of Theorem 6.14 shows that
having 3n - 6 edges in a simple n-vx planar gr. requires 2e = 3f, meaning that every face is
a triangle. If G has some face that is not a triangle, then we can add an edge between non-adjacent
vxs on the boundary of this face to obtain a larger plane graph. Hence the simple plane graphs
with 3n - 6 edges, the triangulations, and the maximal plane graphs are all the same family.
\end{rem}

\subsection{Platonic Solids}

%\begin{defi}
%A polytope is a solid in 3 dimensions with flat faces, straight edges and sharp corners. Faces of a polytope are joined at the edges. A polytope is convex if the line connecting any two points of the polytope lies inside the polytope.
%\end{defi}
\addtocounter{thm}{2}
%\begin{defi}
%A regular or Platonic solid is a convex polytope which satisfies the following:
%\begin{enumerate}
%\item all of its faces are congruent regular polygons,
%\item all vxs have the same \# faces adjacent to them.
%\end{enumerate}
%\end{defi}
\addtocounter{thm}{1}

\begin{cor}
If $K$ is a convex polytope with $v$ vxs, $e$ edges and $f$ faces then $v - e + f = 2$.
\end{cor}

\section{Graph colouring}
\subsection{Vertex colouring}

%\begin{defi}
%A $k$-colouring of $G$ is a labeling $f : V (G) \to \{1, \dots , k\}$. It is a proper $k$-colouring if $(x, y) \in E(G)$ implies $f(x) \ne f(y)$. A gr. $G $ is $k$-colourable if it has a proper $k$-colouring. The
%chromatic number $\chi(G)$ is the minimum $k$ such that $G$ is $k$-colourable. If $\chi(G) = k$, then $G$ is
%$k$-chromatic. If $\chi(G) = k$, but $\chi(H) < k$ for every proper subgr. $H$ of $G$, then $G$ is colour-critical or $k$-critical.
%
%$\chi(K_n) = n$
%\end{defi}
\addtocounter{thm}{2}
\begin{rem}
The vxs having a given colour in a proper colouring must form an independent set, so $\chi(G)$ is the minimum \# independent sets needed to cover $V (G).$ Hence $G$ is $k$-colourable iff $G$ is $k$-partite. Multiple edges do not affect chromatic number. Although we define
$k$-colouring using numbers from $\{1, \dots , k\}$ as labels, the numerical values are usually unimportant, and we may use any set of size $k$ as labels.
\end{rem}

%\subsection{Some motivation}
\addtocounter{subsection}{1}
\addtocounter{thm}{1}\addtocounter{thm}{1}
%\subsection{Simple bounds on the chromatic number}
\subsection{Bounds on $\chi$}
\begin{claim}
If $H$ is a subgr. of $G$ then $\chi(H) \le \chi(G)$.
\end{claim}

\begin{cor}
$\chi(G) \ge \omega(G)$
\end{cor}
\addtocounter{thm}{1}
\begin{pro}
$\chi(G) \ge\frac{|V (G)|}{\alpha(G)}$
\end{pro}

\begin{claim}
For any gr. $G = (V,E)$ and any $U \subseteq V$ we have $\chi(G) \le \chi(G[U]) + \chi(G[V \setminus U])$.
\end{claim}

\begin{claim}
For any graphs $G_1$ and $G_2$ on the same vx set, $\chi(G_1 \cup G_2) \le \chi(G1)\chi(G2).$
\end{claim}


\begin{pro}
\begin{enumerate}[(i)]
\item $\chi(G)\chi(\overline{G})\ge |G|$
\item $\chi(G)+\chi(\overline{G})\le |G|+1$
\end{enumerate}
\end{pro}

\subsection{Greedy colouring}

%\begin{defi}
%The greedy colouring with respect to a vx ordering $v_1, \dots , v_n$ of $V (G)$ is obtained by colouring vxs in the order $v_1, \dots , v_n$, assigning to $v_i$ the smallest-indexed colour not already used on its lower-indexed neighbours.
%\end{defi}
\addtocounter{thm}{3}
%\begin{defi}
%Let $G = (V,E)$ be a graph. We say that $G$ is $k$-degenerate if every subgr. of $G$ has a vx of degree less than or equal to $k$.
%\end{defi}

\begin{pro}
$G$ is $k$-degenerate iff there is an ordering $v_1, \dots , v_n$ of the vxs of $G$ such that each $v_i$ has at most $k$ neighbours among the vxs $v_1, \dots , v_{i-1}$.
\end{pro}

%\begin{defi}
%Define $dg(G)$ to be the minimum $k$ such that $G$ is $k$-degenerate.
%\end{defi}
\addtocounter{thm}{1}
\begin{rem}
$\delta(G) \le dg(G) \le \Delta(G)$.
\end{rem}

\begin{thm}
$\chi(G) \le 1 + dg(G)$
\end{thm}

\begin{cor}
$\chi(G) \le \Delta(G) + 1.$
\end{cor}

\begin{rem}
This bound is tight if $G = K_n$ or if $G$ is an odd cycle.
\end{rem}

\begin{thm}[Brooks 1941]
If $G$ is a conn. gr. other than a clique or an odd cycle, then
$\chi(G) \le \Delta(G).$
\end{thm}

\subsection{Colouring planar graphs}
\begin{claim}
A (simple) planar $G$ contains a vx $v$ of degree at most 5.
\end{claim}

\begin{cor}
A planar $G$ is 5-degenerate and thus 6-colourable.
\end{cor}
\begin{thm}[5 colour theorem, Heawood 1890]
Every planar $G$ is 5-colourable.
\end{thm}

\begin{thm}[Appel-Haken 1977, conjectured by Guthrie in 1852]
Every planar gr. is 4-colourable. (the countries of every plane map can be 4-coloured so that neighbouring countries get
distinct colours).
\end{thm}
\addtocounter{thm}{1}
\subsection{Art gallery thm}

\begin{thm}
For any museum with $n$ walls, $\left\lfloor n/3\right\rfloor$ guards suffice.
\end{thm}

\section{Col. results}
\begin{thm}[Gallai, Roy]
If $D$ is an orientation of $G$ with longest path length $l(D)$, then $\chi(G) \le 1 + l(D)$. Furthermore, equality holds for some orientation of $G$.
\end{thm} 

\subsection{Large girth and large $\chi$}

The bound $\chi(G) \ge \omega(G)$ can be tight, but (surprisingly) it can also be arbitrarily bad. There are graphs having arbitrarily large chromatic number, even though they do not contain $K_3$. Many constructions of such graphs are known, though none are trivial. We give one here.
\addtocounter{thm}{1}
\begin{thm}
Mycielski’s construction produces a $(k + 1)$-chromatic triangle-free gr. from a $k$-chromatic triangle-free graph.
\end{thm}

%\begin{defi}
%The girth of a gr. is the length of its shortest cycle.
%\end{defi}
\addtocounter{thm}{1}
\begin{thm}[Erdos 1959]
Given $k \ge 3$ and $g \ge 3$, there exists a gr. with girth at least g and
chromatic number at least k.
\end{thm} 
\addtocounter{thm}{2}
\begin{thm}
There is a tournament on $n$ vxs where any $\frac{\log_2(n)}{2}$ vxs are beaten by some other vx.
\end{thm}
\subsection{$\chi$ and clique minors}
%\begin{defi}
%Let $e = (x, y)$ be an edge of a gr. $G = (V,E)$. By $G\slash e$ we denote the gr. obtained from $G$ by contracting the edge $e$ into a new vx $v_e$, which becomes adjacent to all the former neighbours of $x$ and of $y$. 
%
%$H$ is a minor of $G$ if it can be obtained from G by deleting vxs and edges, and contracting edges.
%\end{defi}
\addtocounter{thm}{3}
\begin{thm}[Mader]
If the average degree of $G$ is at least $2t-2$ then $G$ has a $K_t$ minor.
\end{thm}

\begin{rem}
It is known that $\overline{d}(G) \ge ct\sqrt{log(t)}$ already implies the existence of a $K_t$ minor in $G$, for some constant $c > 0$.
\end{rem}
\subsection{Edge-colourings}
%\begin{defi}
%A $k$-edge-colouring of $G$ is a labeling $f : E(G) \to [k]$. A proper $k$-edge-colouring is a $k$-edge-colouring such that edges sharing a vx receive different colours, equivalently, each colour class is a matching. A $G$ is $k$-edge-colourable if it has a
%proper $k$-edge-colouring. The edge-chromatic number or chromatic index $\chi'(G)$ is the minimum $k$ such that $G$ is $k$-edge colourable.
%\end{defi}
\addtocounter{thm}{1}
\begin{rem}
\begin{enumerate}[(i)]
\item An edge-colouring of a $G$ is the same as a vx-colouring of its line gr. $L(G)$.
\item A $G$ with maximum degree $\Delta$ has $\chi'(G) \ge \Delta$ since the edges incident to a vx of degree $\Delta$ must have different colours.
\item If $G$ has maximum degree $\Delta$ then $L(G)$ has maximum degree at most $2(\Delta-1).\Rightarrow \chi'(G) \le 2\Delta - 1$ 
\end{enumerate}
\end{rem}

\begin{thm}[König 1916]
If $G$ is a bipartite multigraph, then $\chi'(G) = \Delta(G)$.
\end{thm}

\begin{thm}[Vizing]
Let $G$ be a simple gr. with maximum degree $\Delta$. Then $\Delta(G)\le \chi'(G) \le \Delta(G) + 1.$
\end{thm}
\addtocounter{thm}{1}

\subsection{List colouring}

%\begin{defi}
%For each vx $v$ in a $G$, let $L(v)$ denote a list of colours available for $v$. A list colouring or choice function from a given collection of lists is a proper colouring $f$ such that $f(v)$ is chosen from $L(v)$. A $G$ is $k$-choosable or $k$-list-colourable if it has a proper list colouring from every assignment of $k$-element lists to the vxs. The list chromatic number or choosability $\chi_l(G)$ is the minimum $k$ such that $G$ is $k$-choosable.
%\end{defi}
\addtocounter{thm}{1}
\begin{thm}[Erdos, Rubin, Taylor 1979]
If $m = \binom{2k-1}{k}$, then $K_{m,m}$ is not $k$-choosable.
\end{thm}

%\begin{defi}
%Let $L(e)$ denote the list of colours available for $e$. A list edge-colouring is a proper edge-colouring $f$ with $f(e)$ chosen from $L(e)$ for each $e$. The list chromatic index or edge-choosability $\chi'_l(G)$ is the minimum $k$ such that $G$ has a proper list edge-colouring for each assignment of lists of size $k$ to the edges. Equivalently, $\chi'_l(G) = \chi_l(L(G))$, where $L(G)$ is the line gr. of $G$.
%\end{defi}
\addtocounter{thm}{2}
\begin{thm}[Galvin 1995]
$\chi'_l(K_{n,n}) = n$
\end{thm}

%\begin{defi}
%A kernel of a digr. is an independent set $S$ having an edge to every vx
%outside $S$. A digr. is kernel-perfect if every induced sub-digr. has a kernel. Given a function $f : V (G) \to \mathbb{N}$, the $G$ is $f$-choosable if a proper list colouring can be chosen whenever the lists
%satisfy $|L(x)|\ge f(x)$ for each $x$.
%\end{defi}
\addtocounter{thm}{1}
\begin{lem}
If $D$ is a kernel-perfect orientation of $G$ and $f(x) = d^-_D(x)$ for all $x \in V (G)$, then $G$ is $(1 + f)$-choosable.
\end{lem}

\section{The Matrix Tree Theorem}

\begin{thm}[Cayley's formula]
There are $n^{n-2}$ labeled trees on $n$ vxs.
\end{thm}

Now consider an arbitrary conn. simple $G$ on vx set $[n]$, and denote the \# spanning trees by $t(G)$. The following celebrated result is Kirchhoff’s matrix tree theorem. To formulate it, consider the incidence matrix $B$ of $G$ (as in Def 1.13), and replace one of the
two 1’s by -1 in an arbitrary manner to obtain the matrix $C$ (we say $C$ is the incidence matrix of an orientation of $G$). $ M = CC^T$ is then a symmetric $n \times n$ matrix, which is 

\[
\begin{pmatrix}
d(1) & 0 & 0  \\
0 & \ddots & 0  \\
0 & 0 & d(n)  
\end{pmatrix}-A_G\]

\begin{thm}[Matrix tree theorem]
We have $t(G) = det M_{ii}$ for all $i = 1, \dots , n$, where $M_{ii}$ results
from $M$ by deleting the $i$-th row and the $i$-th column.
\end{thm}

\begin{thm}[Binet,  Cauchy]
If $P$ is an $r \times s$ matrix and $Q$ is an $s \times r$ matrix with $r\le s$, then \[det(PQ) =\sum_Z (det P_Z)(detQ_Z)\]
where $P_Z$ is the $r\times r$ submatrix of $P$ with column set $Z$, and $Q_Z$ is the $r\times r$ submatrix of $Q$ with the corresponding rows $Z$, and the sum is over all $r$-sets $Z \subseteq [s]$.
\end{thm}

\addtocounter{subsection}{1}
%\subsection{Lattice paths and determinants}
%see p.55 and following

\section{More Thms on Hamiltonicity}
%\begin{defi}
%The (Hamiltonian) closure of a $G$, denoted $C(G)$, is the supergr. of $G$ on $V (G)$ obtained by iteratively adding edges between pairs of nonadjacent vxs whose degree sum is at least $n$, until no such pair remains.
%\end{defi}
\addtocounter{thm}{1}
\begin{thm}[Bondy Chvátal 1976]
A simple $n$-vx gr. is Hamiltonian iff its closure is Hamiltonian.
\end{thm}
\begin{thm}[Chvátal 1972]
Suppose $G$ has vx degrees $d_1\le \dots d_n$. If $i < n/2$ implies
that $d_i > i$ or $d_{n-i} \ge n - i$, then $G$ is Hamiltonian.
\end{thm}

\begin{thm}[Chvátal-Erdos 1972]
If $\kappa (G)\ge\alpha(G)$, then $G$ has a Hamiltonian cycle (unless
$G = K_2$).
\end{thm}

\subsection{Pósa’s Lemma}
Let $P$ be a path in a $G$, say from $u$ to $v$. Given a vx $x \in P$, we write $x^-$ for the vx preceding $x$ on $P$, and $x^+$ for the vx following $x$ on $P$ (whenever these exist). Similarly, for $X \subseteq V (P)$ we put $X^\pm:=\{x^\pm:x\in X\}$

If $x \in P\setminus u$ is a neighbour of $u$ in $G$, then $P \cup \{(u, x)\}\setminus\{(x, x^-)\}$ (which is a path in $G$ with vx
set $V (P)$) is said to have been obtained from $P$ by a rotation fixing $v$. A path obtained from $P$ by a (possibly empty) sequence of rotations fixing $v$ is a path derived from $P$. The set of starting vxs of paths derived from $P$, including $u$, will be denoted by $S(P)$. As all paths derived from $P$ have the same vx set as $P$, we have $S(P) \subseteq V (P)$.

\begin{rem}
If some sequence of rotations can delete the edge $(x, x^-)$, call this edge a broken edge. Note that every interval of the original path not containing broken edges is traversed by all derived paths as a whole piece (however, the direction can change).
\end{rem}

%\begin{defi}
%For a $G$ and a subset $S \subseteq V (G)$, let $\partial S = \{v \in G\setminus S : \exists y \in  S, v\sim y\}$.
%\end{defi}
\addtocounter{thm}{1}
\begin{lem}
Let $G$ be a graph, let $P = u \dots v$ be a longest path in $G$, and put $S := S(P)$. Then $\partial S \subseteq S^- \cup  S^+$.
\end{lem}

\begin{lem}
Let $G$ be a graph, let $P = u \dots v$ be a longest path in $G$, and put $S := S(P)$. If $deg(u) \ge 2$ then $ G$ has a cycle containing $S \cup \partial S$.
\end{lem}

\begin{cor}
Fix $k \ge 2$ and let $G$ be a gr. such that for all $S \subseteq V (G)$ with $|S|\le k$, we have $|\partial S|\ge |2S|$. Then $G$ has a cycle of length at least $3k$.
\end{cor}

\subsection{Tournaments}
%\begin{defi}
%A tournament is a directed gr. obtained by assigning a direction to every
%edge of the complete graph. That is, it is an orientation of $K_n$.
%\end{defi}
\addtocounter{thm}{1}
\begin{thm}
Every tournament has a Hamilton path.
\end{thm}

%\begin{defi}
%A tournament is strongly conn. if for all $u, v$ there is a directed path from $u$ to $v$.
%\end{defi}
\addtocounter{thm}{1}
\begin{thm}
A tournament $T$ is strongly conn. iff it has a Hamilton cycle.
\end{thm}

\section{Kuratowski’s Theorem}

%\begin{defi}
%A subdivision of a gr. $H$ is a gr. obtained from $H$ by replacing the edges of $H$ by internally vx disjoint paths of non-zero length with the same endpoints.
%\end{defi}
\addtocounter{thm}{2}
\begin{rem}
If $G$ contains a subdivision of $H$, it also contains an $H$-minor.
\end{rem}

%\begin{defi}
%A Kuratowski gr. is a gr. which is a subdivision of $K_5$ or $K_{3,3}$. If $G$ is a gr. and $H$ is a subgr. of $G$ which is a Kuratowski gr. then we say that $H$ is a Kuratowski subgr. of $G$.
%\end{defi}
\addtocounter{thm}{1}
\begin{thm}[Kuratowski 1930]
A gr. is planar iff it has no Kuratowski subgraph.
\end{thm}

%\begin{defi}
%A straightline drawing of a planar $G$ is a drawing in which every edge is a straight line.
%\end{defi}
\addtocounter{thm}{1}

\begin{thm}
If $G$ is a gr. with no Kuratowski subgr. then $G$ has a straightline drawing in the plane.
\end{thm}

\subsection{Convex drawings of 3-conn. graphs}

%\begin{defi}
%A convex drawing of $G$ is a straightline drawing in which every non-outer face of $G$ is a convex polygon, and the outer face is the complement of a convex polygon. (That is, the boundary of each face is the boundary of a convex polygon).
%\end{defi}
\addtocounter{thm}{1}

\begin{thm}[Tutte 1960]
If $G$ is a 3-conn. gr. which has no Kuratowski subgraphs then
$G$ has a convex drawing in the plane with no three vxs on a line.
\end{thm}

\begin{lem}[Thomassen 1980]
Every 3-conn. $G$ with at least five vxs has an edge
$e$ such that $G\slash e$ is 3-conn.
\end{lem}

\begin{lem}
If $G$ has no Kuratowski subgraphs, then $G\slash e$ has no Kuratowski subgraph, for any edge $e\in E(G)$.
\end{lem}

\subsection{Reducing the general case to the 3-conn. case}

%\begin{defi}
%Given a subdivision $H'$ of $H$, we call the vxs of the original gr. branch vxs.
%\end{defi}
\addtocounter{thm}{2}
\begin{fact}
We make three observations. 
\begin{enumerate}
\item  In a Kuratowski subgraph, there are three internally vx-disjoint paths connecting any two
branch vxs. For $K_5$-subdivisions, we even have four such paths.
\item In a Kuratowski subgraph, there are four internally vx-disjoint paths between any two pairs
of branch vxs.
\item Any cycle in a subdivision contains at least three branch vxs.
\end{enumerate}
\end{fact}

\begin{pro}
Let $G$ be a gr. with at least 4 vxs which has no Kuratowski subgraph, and suppose that adding an edge joining any pair of non-adjacent vxs creates a Kuratowski subgraph.
Then G is 3-conn.
\end{pro}

\section{Ramsey Theory}

\begin{pro}
Among six people it is possible to find three mutual acquaintances or three mutual non-acquaintances.
\end{pro}

As we shall see, given a natural number $s$, there is an integer $R$ such that if $n \ge R$ then every colouring of the edges of $K_n$ with red and blue contains either a red $K_s$ or a blue $K_s$. More generally, we define the Ramsey number $R(s, t)$ as the smallest value of $N$ for which every red-blue colouring of $K_N$ yields a red $K_s$ or a blue $K_t$. In particular, $R(s, t) = \infty$ if there is no such $N$ such that in every red-blue
colouring of $K_N$ there is a red $K_s$ or a blue $K_t$. It is obvious that
$ R(s, t) = R(t, s)$ for every $s, t \ge 2$ and $R(s, 2) = R(2, s) = s$. 

\begin{thm}[Erdös, Szekeres]
The function $R(s, t)$ is finite for all $s, t \ge 2$. Quantitatively, if $s > 2 $ and $t > 2$ then $R(s, t)\le R(s - 1, t) + R(s, t - 1)$ and $R(s, t) \le \binom{s+t-2}{s-1}$.
\end{thm}

\begin{thm}
Given $k$ and $s_1, s_2, \dots , s_k$, if $N$ is sufficiently large, then every colouring of $K_N$ with $k$ colours is such that for some $i, 1 \le i \le k$, there is a $K_{s_i}$ coloured with the $i$-th colour. The
minimal value of $N$ for which this holds is usually denoted by $R_k(s_1, \dots , s_k)$ , and it satisfies $R_k(s_1, \dots , s_k)\le R_{k-1}(R(s_1, s_2), s_3, \dots , s_k)$.
\end{thm}

\begin{thm}
Let $min\{s, t\} > 3$. Then
$R^{(3)}(s, t) \le R(R^{(3)}(s - 1, t),R^{(3)}(s, t - 1)) + 1$
\end{thm}

\subsection{Applications}

\begin{thm}[Erdos-Szekeres 1935]
Given an integer $m$, there exists a (least) integer $N(m)$ such
that every set of at least $N(m)$ points in the plane, with no three collinear, contains an $m$-subset forming a convex $m$-gon.
\end{thm}
\subsection{Bounds on Ramsey numbers}

\begin{thm}[Erdös 1947]
For $p \ge 3$, we have $R(p, p) > 2^{p/2}$.
\end{thm}

\begin{thm}
We have $R_k(3)\stackrel{def}{=}R_k(3,\dots,3)\le\left\lfloor e\cdot k!\right\rfloor+1$.
\end{thm}

\subsection{Ramsey theory for integers}

\begin{thm}[Schur 1916]
For every $k \ge 1$ there is an integer $m$ such that every $k$-colouring of [m] contains integers $x, y, z$ of the same colour such that $x + y = z.$
\end{thm}

\subsection{Graph Ramsey numbers}
%\begin{defi}
%Let $G_1,G_2$ be graphs. $R(G_1,G_2)$ is the minimal $N$ such that any red/blue colouring of $K_N$ contains either a red copy of $G_1$, or a blue copy of $G_2$.
%\end{defi}
\addtocounter{thm}{1}
\begin{rem}
Note that $R(G1,G2) \le R(|G1|,|G2|)$.
\end{rem}

\begin{thm}[Chvatal 1977]
If $T$ is any $m$-vx tree, then $R(T,K_n) = (m - 1)(n - 1) + 1$
\end{thm}

\section{Extremal problems}
\addtocounter{thm}{1}
%\begin{defi}
%$ex(n,H)$ is the maximal value of $e(G)$ among graphs $G$ with $n$ vxs containing no $H$ as a subgraph.
%\end{defi}

\addtocounter{thm}{2}
\subsection{Turán’s theorem}
%\begin{defi}
%We call the gr. $K_{n_1,\dots,n_r}$ with $|n_i - n_j|\le 1$ the Turán graph, denoted by $T_{n,r}$.
%\end{defi}
\addtocounter{thm}{1}
\begin{thm}[Turan 1941]
Among all the $n$-vx simple graphs with no $(r + 1)$-clique, $T_{n,r}$ is
the unique gr. having the maximum \# edges.
\end{thm} 

\begin{que}
Let $a_1, \dots , a_n \in \mathbb{R}^d$ be vectors such that $|a_i|\ge 1$ for each $i \in [n]$. What is the maximum \# pairs satisfying $|a_i + a_j| < 1$?
\end{que}

\begin{claim}
There are at most $\left\lfloor \frac{n^2}{4}\right\rfloor$ such pairs.
\end{claim}

%\begin{defi}
%For some fixed gr. $H$, we define $\pi(H) = \lim\limits_{n\to\infty} ex(n,H)/\binom{n}{2}$
%\end{defi}
\addtocounter{thm}{1}
\begin{thm}[Erdos-Stone]
Let $H$ be a gr. of chromatic number $\chi(H) = r + 1$. Then for
every $\varepsilon > 0$ and large enough $n$, $\left(1-\frac{1}{r}\right)\frac{n^2}{2}\le ex(n,H)\le\left(1-\frac{1}{r}\right)\frac{n^2}{2}+n^2\varepsilon$
\end{thm}
 \addtocounter{thm}{1}
\subsection{Bipartite Turán Theorems}

\begin{thm}
If a $G$ on n vxs contains no 4-cycles, then $e(G)\le\left\lfloor\frac{n}{4}(1+\sqrt{4n-3})\right\rfloor$.
\end{thm}
\addtocounter{thm}{1}
\begin{thm}[Kovári-Sós-Turán]
For any integers $r \le s$, there is a constant $c$ such that every
$K_{r,s}$-free gr. on $n$ vxs contains at most $cn^{1-\frac{1}{r}}$ edges. In other words, $ex(n,K_{r,s}) \le cn^{1-\frac{1}{r}}$ 
\end{thm}

\begin{thm}
There is $c$ depending on $k$ such that if $G$ is a gr. on $n$ vxs that contains no copy of $C_{2k}$, then $G$ has at most $cn^{1+\frac{1}{k}}$ edges.
\end{thm}

\begin{que}
Given $n$ points in the plane, how many pairs can be at distance 1?
\end{que}

\begin{thm}[Erdos]
There are at most $cn^{3/2}$ pairs.
\end{thm}

\section{Exercises}
\subsection{Assignment 1}


\addtocounter{problem}{1}
%\begin{problem}
%Given a $G$ with vx set $V = \{v_1,\dots,v_n\}$ we define the degree sequence of $G$ to be the list $d(v_1),\dots, d(v_n)$ of degrees in decreasing order.
%\end{problem}

\addtocounter{problem}{1}

\begin{problem}
Prove that if a $G$ is not conn. then its complement $\overline{G}$ is conn. Converse is not true.
\end{problem}

\begin{problem}
Show that every gr. on at least two vxs contains two vxs of equal
degree.
\end{problem}

\begin{problem}
Prove that every gr. with $n \ge 7$ vxs and at least $5n- 14$ edges contains a subgr. with minimum degree at least 6.
\end{problem}

\begin{problem}
Show that in a conn. gr. any two paths of maximum length share at least
one vx.
\end{problem}

\begin{problem}
Prove that a gr. is bipartite iff it contains no cycle of odd length.
\end{problem}

\subsection{Assignment 2}

\begin{problem}
Show that in a tree containing an even \# edges, there is at least one
vx with even degree.
\end{problem}

\begin{problem}
Given a $G$ and a vx $v \in V (G)$, $G - v$ denotes the subgr. of $G$ induced by the vx set $V (G)\setminus\{v\}$. Show that every conn. $G$ of order at least two contains vxs $x$ and $y$ such that both $G- x$ and $G-y$ are conn.
\end{problem}

\begin{problem}
Let $T$ be an $n$-vx tree with exactly $2k$ odd-degree vxs. Prove that $T$ decomposes into $k$ paths (i.e. its edge-set is the disjoint union of $k$ paths).
\end{problem}

\begin{problem}
Prove that a conn. $G$ is a tree iff any family of pairwise (vx-)intersecting paths $P_1,\dots,P_k$ in $G$ have a common vx.
\end{problem}

\begin{problem}
\begin{enumerate}[(a)]
\item Prüfer codes corresponding to stars (i.e. to trees isomorphic to $K_{1,n-1}$) = 1 value.
\item Prüfer codes containing exactly 2 different values = 2 connected stars
\end{enumerate}
\end{problem}

\begin{problem}
Let $T$ be a forest on vx set $[n]$ with components $T_1,\dots,T_r$. Prove, by induction on $r$, that the \# spanning trees on $[n]$ containing $T$ is $n^{r-2}\prod_{i=1}^{r}|T_i|$. Deduce Cayley's formula.
\end{problem}

\subsection{Assignment 3}

\begin{problem}
Prove that a conn. $G$ is $k$-edge-conn. iff each block of $G$ is $k$-edge-conn.
\end{problem}

\begin{problem}
Let $G$ be a gr. and suppose some two vxs $u, v \in V (G)$ are separated by $X \subseteq V (G)\setminus\{u,v\}$. Show that $X$ is a minimal separating set (i.e. there is no proper subset $Y\subset X$ that separates $u$ and $v$) iff every vx in $X$ has a neighbor in the component
of $G-X$ containing $u$ and another in the component containing $v$.
\end{problem}

\begin{problem}
Show that if $G$ is a gr. with $|V (G)| = n \ge k + 1$ and $\delta(G) \ge(n + k-2)/2$ then $G$ is $k$-conn.
\end{problem}

\begin{problem}
Prove that a $G$ with at least 3 vxs is 2-conn. iff for any three vxs $x, y, z$ there is a path from $x$ to $z$ containing $y$.
\end{problem}

\begin{problem}
Let $G$ be a $k$-conn. graph, where $k \ge 2$. Show that if $|V (G)| \ge 2k$ then $G$ contains a cycle of length at least $2k$.
\end{problem}

\subsection{Assignment 4}

\begin{problem}
Show that if $k > 0$ then the edge set of any conn. gr. with $2k$ vxs of odd degree can be split into $k$ trails.
\end{problem}

\begin{problem}
Let $G$ be a conn. gr. that has an Euler tour. T / F?
\begin{enumerate}[(a)]
\item If $G$ is bipartite then it has an even \# edges. T
\item If $G$ has an even \# vxs then it has an even \# edges. F
\item For edges $e$ and $f$ sharing a vx, $G$ has an Euler tour in which $e$ and $f$ appear consecutively. F
\end{enumerate}
\end{problem}

\begin{problem}
Let $G$ be a conn. gr. on n vxs with minimum degree $\delta$. Show that 
\begin{enumerate}[(a)]
\item if $\delta \le \frac{n-1}{2}$ then $G$ contains a path of length $2\delta$, and
\item if $\delta \ge \frac{n-1}{2}$ then $G$ contains a Hamiltonian path.
\end{enumerate}
\end{problem}

\begin{problem}
Show that the maximum \# edges in a non-Hamiltonian gr. on $n \ge 3$
vxs is $\binom{n-1}{2}+1$.
\end{problem}

\subsection{Assignment 5}

\begin{problem}
Let $G$ be a conn. gr. on more than 2 vxs such that every edge is
contained in some perfect matching of $G$. Show that $G$ is 2-edge-conn.
\end{problem}

\begin{problem}
\begin{enumerate}[(a)]
\item Let $G$ be a gr. on $2n$ vxs that has exactly one perfect matching. Show that $G$ has at most $n^2$ edges.
\item Construct such a $G$ containing exactly $n^2$ edges for any $n \in N$.
\end{enumerate}
\end{problem}

\begin{problem}
Let $A$ be a finite set with subsets $A_1,\dots, A_n$, and let $d_1,\dots, d_n$ be positive integers. Show that there are disjoint subsets $D_k \subseteq A_k$ with $|D_k| = d_k$ for all $k \in [n]$ if and
only if $|\bigcup_{i\in I}A_i|\ge \sum_{i\in I}d_i$.
\end{problem}

\begin{problem}
Suppose $M$ is a matching in a bipartite gr. $G = (A\cup B,E)$. We say that a path $P = a_1b_1\dots a_kb_k$ is an augmenting path in $G$ if $b_ia_{i+1}\in M$ for all $i\in [k - 1]$ and $a_1$ and $b_k$ are not covered by $M$. The name comes from the fact that the size of $M$ can be increased
by 
flipping the edges along $P$ (in other words, taking the symmetric difference of $M$ and $P$): by deleting the edges $b_ia_{i+1}$ from $M$ and adding the edges $a_ib_i$ instead.
\begin{enumerate}[(a)]
\item Prove Hall's theorem by showing that if Hall's condition is satisfied and M does not cover $A$, then there is an augmenting path in $G$.
\item Show that if $M$ is not a maximum matching (i.e. there is a larger matching in $G$) then the gr. contains an augmenting path. Is this true for non-bipartite graphs as well? Y
\end{enumerate}
\end{problem}

\begin{problem}
Show that for $k\ge 1$, every $k$-regular ($k- 1$)-edge-conn. gr. on an even \# vxs contains a perfect matching.
\end{problem}

\subsection{Assignment 6}

\addtocounter{problem}{1}
%\begin{problem}
%Determine all positive integers $r$ and $s$, with $r \le s$, for which $K_{r,s}$ is planar.
%\end{problem}

\begin{problem}
\begin{enumerate}[(a)]
\item Show that every planar gr. has a vx of degree at most 5. Is there a planar gr. with minimum degree 5? Y
\item Show that any planar bipartite gr. has vx of degree at most 3. Is there a planar bipartite gr. with minimum degree 3? Y
\end{enumerate}
\end{problem}

\begin{problem}
Show that a conn. plane $G$ is bipartite iff all its faces have even length.
\end{problem}

\begin{problem}
Let $G$ be a gr. on $n\ge 3$ vxs and $3n- 6 + k$ edges for some $k > 0$. Show that any drawing of $G$ in the plane contains at least $k$ crossing pairs of edges.
\end{problem}

\begin{problem}
Let $G$ be a plane gr. with triangular faces and suppose the vxs are colored arbitrarily with three colors. Prove that there is an even \# faces that get all three colors.
\end{problem}

\begin{problem}
Let $S$ be a set of $n\ge 3$ points in the plane such that any two of them have distance at least 1. Show that there are at most $3n - 6$ pairs of distance exactly 1.
\end{problem}

\subsection{Assignment 7}

\begin{problem}
T / F?
\begin{enumerate}[(a)]
\item If $G$ and $H$ are graphs on the same vx set, then $dg(G \cup H)\le dg(G) + dg(H)$. F
\item If $G$ and $H$ are graphs on the same vx set, then $\chi(G \cup H) \le \chi(G) + \chi(H)$. F
\item Every $G$ has a $\chi(G)$-coloring where $\alpha(G)$ vxs get the same color. F
\end{enumerate}
\end{problem}

\begin{problem}
$G$ has the property that any two odd cycles in it intersect (they share at least one vx in common). Prove that $\chi(G) \le 5.$
\end{problem}

\begin{problem}
For a vx $v$ in a conn. $G$, let $G_r$ be the subgr. of $G$ induced by the vxs at distance $r$ from $v$. Show that $\chi(G) \le max_{0\le r\le n}\chi(G_r) + \chi(G_{r+1})$.
\end{problem}

\begin{problem}
Let $l$ be the length of the longest path in a $G$. Prove $\chi(G)\le l + 1$ using the fact that if a gr. is not $d$-degenerate then it contains a subgr. of minimum degree at least $d + 1$.
\end{problem}

\begin{problem}
Suppose the complement of $G$ is bipartite. Show that $\chi(G) = \omega(G).$
\end{problem}

\subsection{Assignment 8}

\begin{problem}
For a given natural number $n$, let $G_n$ be the following gr. with $\binom{n}{2}$ vxs and $\binom{n}{3}$ edges: the vxs are the pairs $(x, y)$ of integers with $1\le x < y \le n$, and for each triple $(x, y, z)$ with $1\le x < y < z\le n$, there is an edge joining vx $(x, y)$ to vx $(y, z)$. Show that for any natural number $k$, the gr. $G_n$ is triangle-free and has chromatic number $\chi(G_n) > k$ provided $n > 2k$.
\end{problem}

\begin{problem}
Show that the theorem of Mader implies the following weakening of Hadwiger's
conjecture: Any $G$ with $\chi(G) \ge 2^{t-2} + 1$ has a $K_t$-minor.
\end{problem}

\begin{problem}
Find the edge-chromatic \# $K_n$ (don't use Vizing's theorem). $n$ for $n$ odd, $n-1$ for $n$ even
\end{problem}

\begin{problem}
Let $G$ be a conn. $k$-regular bipartite gr. with $k\ge 2$. Show, using König's theorem, that $G$ is 2-conn.
\end{problem}

\subsection{Assignment 9}

\begin{problem}
Prove that every $G$ of maximum degree $\Delta$ has an equitable $(\Delta + 1)$-edge-coloring, i.e. one where each color class contains $\left\lfloor e=(\Delta + 1)\right\rfloor$ or $\left\lfloor e=(\Delta + 1)\right\rfloor$ edges, where $e$ is the \# edges in $G$.
\end{problem}

\begin{problem}
The cartesian product $H \times G $ of graphs $H$ and $G$ is the gr. with vx set $V (H)\times V (G)$, with an edge between $(v, u)$ and $(v', u')$ if $v = v'$ and $u$ is adjacent to $u'$ in $G$, or
if $u = u'$ and $v$ is adjacent to $v'$ in $H$. Prove that if $\chi'(H) = \Delta(H)$ then $\chi'(H\times G) = \Delta(H\times G)$
\end{problem}

\begin{problem}
Show that $\chi(C_n) = \chi_l(C_n)$ for any $n \ge 3$.
\end{problem}

\begin{problem}
Let $G$ be a bipartite gr. on $n$ vxs. Prove that $\chi_l(G)\le 1 + \log_2(n)$ using the probabilistic method.
\end{problem}

\begin{problem}
Let $G$ be a complete $r$-partite gr. with all parts of size 2. (In other words, $G$ is $K_{2r}$ minus a perfect matching.) Show, using a combination of induction and Hall's theorem, that $\chi_l(G) = r$.
\end{problem}

\subsection{Assignment 10}

\begin{problem}
How many spanning trees does $K_{r,s}$ have? $r^{s-1}s^{r-1}$
\end{problem}

\begin{problem}
Find the \# spanning trees of $K_n- e$ (the complete gr. on n vxs with one edge removed): $(n-2)n^{n-3}$
%\begin{enumerate}[(a)]
%\item using the Matrix Tree Theorem, and 
%\item using a double counting argument.
%\end{enumerate} 
\end{problem}

\begin{problem}
In this exercise we prove the following alternative form of the matrix-tree theorem. For an $n$-vx conn. $G$, the \# spanning trees in $G$ is equal to the product of the  nonzero eigenvalues of the Laplacian matrix $M$ of $G$, divided by $n$. (This matrix $M$ is as in the lecture notes).
\end{problem}

\begin{problem}
\begin{enumerate}[(a)]
\item Prove that any $n$-by-$n$ bipartite gr. with minimum degree $\delta > n/2$ contains a Hamilton cycle.
\item  Show that this is not necessarily the case if $\delta\le n/2$.
\end{enumerate}
\end{problem}

\subsection{Assignment 11}

\begin{problem}
T / F?
\begin{enumerate}[(a)]
\item If every vx of a tournament has positive in- and out-degree, then the tournament contains a directed Hamilton cycle. F
\item If a tournament has a directed cycle, then it has a directed triangle. T
\end{enumerate} 
\end{problem}

\begin{problem}
Let $G$ be a gr. on $n \ge 3$ vxs with at least $\alpha(G)$ vxs of degree $n- 1$. Show that $G$ is Hamiltonian.
\end{problem}

\begin{problem}
Suppose $G$ is a gr. on $n$ vxs where all the degrees are at least $\frac{n+q}{2}$. Show that any set $F$ of $q$ independent edges is contained in a Hamiltonian cycle.
\end{problem}

\subsection{Assignment 12}

\begin{problem}
The lower bound for $R(p, p)$ that you learn in the lectures is not a constructive proof: it merely shows the existence of a red-blue coloring not containing any monochromatic copy of $K_p$ by bounding the \# bad graphs. Give an explicit coloring on $K_{(p-1)^2}$ that proves $R(p, p) > (p - 1)^2$.
\end{problem}

\begin{problem}
Prove that for every fixed positive integer $r$, there is an $n$ such that any coloring of all the subsets of $[n] $ using $r$ colors contains two non-empty disjoint sets $X$ and $Y$ such that $X, Y $ and $X \cup Y$ have the same color.
\end{problem}

\begin{problem}
Prove that for every $k \ge 2$ there exists an integer $N$ such that every coloring of $[N]$ with $k$ colors contains three distinct numbers $a, b, c$ satisfying $ab = c$ that have the same color.
\end{problem}

\begin{problem}
For every $k\ge 2$ there is an $N$ such that any $k$-coloring of $[N]$ contains three distinct integers $a, b, c$ of the same color satisfying $a + b = c.$
\end{problem}

\begin{problem}
\begin{enumerate}[(a)]
\item Let $n\ge 1$ be an integer. Show that any sequence of $N\ge R(n, n)$  distinct numbers, $a_1,\dots, a_N$ contains a monotone (increasing or decreasing) subsequence of length $n$.
\item Let $k, l \ge1$ be integers and show that any sequence of $kl+1$ distinct numbers $a_1,\dots, a_{kl+1}$ contains a monotone increasing subsequence of length $k + 1$ or a monotone decreasing
subsequence of length $l + 1$.
\end{enumerate}
\end{problem}

\subsection{Assignment 13}

\begin{problem}
Let $H$ be an arbitrary fixed gr. and prove that the sequence $ex(n,H)/\binom{n}{2}$ is (not necessarily strictly) monotone decreasing in $n$.
\end{problem}

\begin{problem}
Among all the $n$-vx $K_{r+1}$-free graphs, the Turan gr. $T_{n,r}$ contains the maximum \# triangles (for any $r, n \ge 1$).
\end{problem}

\begin{problem}
Let $X$ be a set of $n$ points in the plane with no two points of distance greater than 1. Show that there are at most $\frac{n^2}{3}$ pairs of points in $X$ that have distance greater than $\frac{1}{\sqrt{2}}$.
\end{problem}
\section{Notation}



\setbox\ltmcbox\vbox{
\makeatletter\col@number\@ne
\begin{longtable}{cl}
$V(G)$&vxs of $G$\\
$E(G)$&edges of $G$\\
$e(G)$&\# edges\\
$A(G)$&adjacency matrix\\
$B(G)$&incidence matrix\\
$N(v)$&neighbourhood\\
$d(v)$&degree\\
$\delta(G)$&min $d(v)$\\
$\Delta(G)$&max $d(v)$\\
$\overline{d}(G)$&average $d(v)$\\
$\overline{G}$&complement of $G$\\
$C_k$&complete gr on $k$ vxs\\
$C_{r,s}$&compl. bipatite gr.\\
$\omega(G)$&max clique\\
$\alpha(G)$&max ind. set\\
$\kappa(G)$&vx-connectivity\\
$\kappa'(G)$&edge-conn.\\
$d(u,v)$&distance\\
$L(G)$&line graph\\
$\nu(G)$&max matching\\
$\tau(G)$&min vx cover\\
$\xi(G)$&chrom nr\\
$dg(G)$&min degeneracy\\
$G/e$&$e$ contracted\\
$G-e$&$e$ deleted\\
$\xi'(G)$&edge chrom nr\\
$L(v)$&list of col avail\\
$\xi_l(G)$&list chrom nr\\
$R(s,t)$&Ramsey nr\\
$R(G_1,G_2)$&graph Ramsey nr\\
$ex(n,H)$&extremal number\\
$T_{n,r}$&Turan graph\\
$\pi(H)$&$\lim\limits_{n\to\infty}ex(n,H)(\binom{n}{2})$\\
\end{longtable}
\unskip
\unpenalty
\unpenalty}

\unvbox\ltmcbox

%\end{tiny}
\end{multicols}
\end{document}